\documentclass[a4paper,12pt]{report}
\usepackage{graphicx} % inserting images
%\usepackage{hyperref} % PDF links
\usepackage{footnote}
\usepackage{threeparttable}
\usepackage{setspace} % for switching between double/single space in document
\usepackage{fancyhdr} % package for changing Headings style
%\usepackage{endnotes}
% setting the margins of page
%\usepackage[top=3cm,right=3cm,bottom=2.5cm,left=2.5cm]{geometry}    
%\usepackage[pagebackref,colorlinks,citecolor=blue,linkcolor=black]{hyperref}   
\usepackage[colorlinks,citecolor=blue,linkcolor=black]{hyperref}  
\usepackage{algorithmic}
\usepackage{color, colortbl}
\usepackage{subfig}
%\usepackage{subfigure}
\usepackage{amsmath,graphicx}
\usepackage{amssymb}
\usepackage{multirow}
\usepackage{multicol}
\usepackage{array}
\usepackage{tabularx}
%\usepackage{backref}
\usepackage{amsthm}
\usepackage{pseudocode}
\usepackage{makeidx}
\usepackage{enumitem}
\setlist[itemize,1]{label={\fontfamily{cmr}\fontencoding{T1}\selectfont\textbullet}}
\usepackage{endnotes}
\usepackage{float}
\usepackage{etoolbox}
\usepackage[extrafootnotefeatures]{xepersian}
\usepackage[bottom]{footmisc}

%\let\LTRfootnote=\endnote
%\usepackage{glossaries}

%\def\backrefpagesname{page(s)}
\newtheorem{definition}{تعریف}[section]
%\usepackage[utf8x]{inputenc}
%\DeclareUnicodeCharacter{200F}{\nobreakspace}%Nonbreaking-space
\makeindex
%\makeglossaries

%\let\LTRfootnote=\endnote
%\renewcommand{\notesname}{\vspace{-16pt}}
%%%%         

% tell tex engine address of folder containing your pictures
\graphicspath{{images/}}
%\let\footnote=\endnote
\DeclareMathOperator*{\argmax}{arg\,max}

\DeclareMathOperator*{\argmin}{arg\,min}
%%%
% commands to print the page number in header
\pagestyle{fancy}
\cfoot{}
\lhead{\thepage}

% commands related to XePersian package
%\settextfont[Scale=1.1]{XB Zar}
%\usepackage{xepersian}
%\settextfont[Scale=1.1]{B Nazanin}

%%%%%%%%%%%%%%%%%%%%%%%%%%
% چنانچه می‌خواهید اعداد در فرمول‌ها، انگلیسی باشد، خط زیر را غیرفعال کنید
%\setdigitfont[Scale=1.1]{B Nazanin}
\setlatintextfont{Times New Roman}
\definecolor{Gray}{gray}{0.9}
\definecolor{LightCyan}{rgb}{0.88,1,1}

% -------------------------------------

\newcommand{\fatitle}{یک مدل پیش‌بینی خطا با استفاده از  معیارهای جهش}
\newcommand{\faAuthor}{علی محبی}
\newcommand{\stdnum}{95210679}
\newcommand{\fasupervisor}{دکتر حسن میریان }
\newcommand{\fadate}{دی 96}
\newcommand{\famajor}{گرایش نرم‌افزار}
\newcommand{\falevel}{کارشناسی ارشد}
\newcommand{\fadepart}{دانشکده مهندسی کامپیوتر}

\newcommand{\entitle}{A Model for Bug Prediction Using Mutation Metrics }

\newcommand{\momtahenin}{دکتر مهدیه سلیمانی باغشاه}
% در این فایل، دستورها و تنظیمات مورد نیاز، آورده شده است.
%-------------------------------------------------------------------------------------------------------------------
%\usepackage[top=40mm, bottom=40mm, left=25mm, right=35mm]{geometry}
\usepackage{footnote}
\usepackage{threeparttable}
\usepackage{setspace} % for switching between double/single space in document
%\usepackage{endnotes}
% setting the margins of page  
%\usepackage[pagebackref,colorlinks,citecolor=blue,linkcolor=black]{hyperref}    
\usepackage{algorithmic}
\usepackage{color, colortbl}
\usepackage{subfig}
%\usepackage{subfigure}
\usepackage{multirow}
\usepackage{multicol}
\usepackage{array}
\usepackage{tabularx}
%\usepackage{backref}
\usepackage{pseudocode}

% بسته‌ای برای تنطیم حاشیه‌های بالا، پایین، چپ و راست صفحه


% فراخوانی بسته زی‌پرشین و تعریف قلم فارسی و انگلیسی
\usepackage{xepersian}
\settextfont[Scale=1.1]{B Nazanin}

%%%%%%%%%%%%%%%%%%%%%%%%%%
% چنانچه می‌خواهید اعداد در فرمول‌ها، انگلیسی باشد، خط زیر را غیرفعال کنید
\setdigitfont[Scale=1.1]{B Nazanin}
%%%%%%%%%%%%%%%%%%%%%%%%%%

\addtolength{\topmargin}{-1.3cm}
\addtolength{\textheight}{2.1cm}
\addtolength{\oddsidemargin}{-0.7cm}
%\addtolength{\evensidemargin}{-1cm}
\addtolength{\textwidth}{2.1cm}

%\renewcommand{\baselinestretch}{1.2}
\begin{document}

\setcounter{secnumdepth}{3}
\begin{center}
\thispagestyle{empty}
\includegraphics{logo} \\
\vskip 1cm
%\begin{Large}

‎\textbf{‎دانشگاه صنعتی شریف}\\
‎\textbf{‎\fadepart}\\ 
‎\textbf{سمینار \falevel \famajor}‎
\vskip 1cm
‎\textbf{‎\large{عنوان:}}\\ \textbf{\fatitle}\\\textbf{‎\lr{\entitle}}

\vskip 1cm
‎\textbf{نگارش:}‎\\ ‎\textbf{\faAuthor}‎\\ ‎\textbf{\stdnum}‎
\vskip 1cm
‎\textbf{استاد راهنما:}‎\\ ‎\textbf{\fasupervisor}‎
\vskip 1cm
‎\textbf{‎استاد ممتحن داخلی:}\\ ‎\textbf{\momtahenin}
\vskip 4cm
‎‎\fadate
%‎\end{Large}‎
\end{center}


%\include{IRTC}

\pagenumbering{arabic}
\settextfont[Scale=1.1]{B Nazanin}



%\chapter{راهنمای استفاده از کلاس}
\thispagestyle{empty}
\noindent
\renewcommand*\thesection{\arabic{section}}
\textbf{\large{چکیده:}}
تکنیک‌های پیش بینی خطا می‌تواند تیم‌های تضمین کیفیت را یاری دهد تا تلاش خود را متمرکز قسمت‌های حاوی خطا کنند. با استفاده از ابزارهای آماری مدرن  ساخت و به کارگیری مدل‌های پیش بینی سریع و آسان شده است. پژوهشگران سعی دارند تا با بهبود قسمت‌های مختلف پیش بینی خطا مدلهای کاراتری را تولید کنند. تعدادی از این تلاشها عبارتند از: ‌تولید مجموعه داده‌های دقیقتر، بکارگیری روشهای نوین یادگیری ماشین و ارائه‌ی معیارهای جدید. معیارهای پیش بینی از اصلی ترین اجزای مدل پیش بینی است که بیشترین تاثیر را بر نتیجه دارد. در این گزارش با‌ ارائه‌ی مدلهای مبتنی بر جهش و فرآیند سعی در ارتقای کارایی پیش بینی دارد. 
\\\\
\textbf{واژه‌های کلیدی:}\textit{ 
پیش بینی خطا، معیارهای نرم افزار، آزمون نرم افزار، جهش
}\\\\
\section{مقدمه}
\label{sec:intro}
  توسعه‌دهندگان نرم‌افزار از طریق گزارش خطا در سیستم‌های ردگیری مشکلات \endnote{Issue Tracking Systems} و یا شکست در آزمون نرم‌افزار متوجه حضور خطا می‌شوند و پس از آن به جستجوی محل خطا و درک مشکل  نرم‌افزار می‌پردازند. کشف زود هنگام خطاها موجب صرفه جویی در زمان و هزینه می‌شود و فرآیند اشکال زدایی را تسهیل می‌بخشد. روش‌های پیش بینی خطا با به کارگیری تکنیک‌های یادگیری ماشین مانند دسته ‌بندهای \lr{Navie Bayes } و \lr{Logistic Regression}  و غیره، مدلی مناسب ارائه می‌دهند و وجود خطا را در یک موجودیت که می‌تواند فایل و یا تابع باشد پیش‌بینی می‌کنند. داده‌های مورد نیاز مانند کد منبع و گزارش خطا، به منظور آموزش مدل از سیستم‌های ردگیری مشکلات و مخازن منابع کد \endnote{Source Code Repositories} استخراج می‌شوند. \\
  
 به منظور ساخت یک مدل یادگیری ماشین لازم است که از داده‌های موجود ویژگی\endnote{Feature} و یا صفاتی \endnote{Attributes}  برای هر موجودیت استخراج شوند. این ویژگی ها به همراه برچسب سالم و یا خطا دار بودن موجودیت جهت آموزش مدل یادگیری بکار می‌رود.  ویژگی  در پژوهشهای حوزه‌ی پیش بینی خطا معیار \endnote{Metric}  نامیده می‌شود.  معیارها را می‌توان به دو دسته‌ی کلی کد منبع و فرآیند تقسیم کرد. در پژوهش‌های قبلی نشان داد شده که در بسیاری از موارد معیارهای مبتنی بر فرآیند بهتر از سایر معیارها عمل می‌کنند\cite{rahman2013and}\cite{radjenovic2013software}. \\
 
  یکی از مشکلات این روش‌ها عدم دقت کافی در پیش بینی خطا به علت محدودیت‌های عملکردی مدل‌های یادگیری می‌باشد. پژوهشگران برای بهبود پیش بینی خطا سعی کرده‌اند که معیارهای جدید ارائه دهند تا داده‌های بیشتر و موثرتری فراهم شود. در برخی پژوهش‌ها از مدل بدست آمده در پروژه‌های دیگر در پیش بینی خطای یک پروژه استفاده شده تا کمبود داده‌های آموزش جبران شود و یا تکنیک‌های یادگیری ماشین به روزتری مانند یادگیری گروهی\endnote{Ensmble Learning} به کار گرفته شده است. در پژوهش‌های اخیر  با استفاده از موارد آزمون و مفاهیم آزمون جهش معیارهایی مانند تعداد جهش یافته‌هایی که توسط آزمون‌ها پوشش داده شده مطرح  شده است و در ساخت مدل پیش‌بینی خطا \cite{bowes2016mutation} و مکان‌یابی خطا  \cite {papadakis2015metallaxis}استفاده شده است. در این پژوهش‌ها معیارهای مطرح شده تنها در یک مقطع زمانی در نظر گرفته شده و مدل پیش بینی با ترکیب این معیارها با معیارهای کد منبع ساخته شده است.\\
  
 در این پژوهش معیارهای جهش ارائه شده در پژوهش‌های قبلی در کنار معیارهای فرآیند قرار می‌گیرند و تاثیر ترکیب این دو نوع از معیار در کارایی مدل پیش بینی حاصل ارزیابی می‌شود. همچنین معیارهای جهش جدیدی با در نظر گرفتن تاریخچه‌ی نرم افزار ارائه می‌گردد و یک مدل پیش بینی با به کارگیری معیارهای فرآیند و معیارهای ارائه شده ساخته می‌شود و میزان بهبود پیش بینی خطا ارزیابی می‌شود.  

\section{بررسی کارهای مرتبط پیشین}
\label{sec:survey}

\subsection{پیش بینی خطا}
\subsubsection{معیارهای ارزیابی}
\label{subsec:eval}
معیارهای ارزیابی را می‌توان به دسته‌ی کلی  معیارهای دسته بندی و رگرسیون تقسیم کرد.  معیارهای دسته بندی را می‌توان با استفاده از ماتریس درهم‌ریختگی\LTRfootnote{Confusion Matrix} محاسبه نمود. در ماتریس درهم ریختگی پیش‌بینی خطا، عناصر  به صورت زیر تعریف می‌شوند.  همچنین نحوه‌ی محاسبه‌ی معیارها در جدول \ref{tab:eval-metircs} آمده است. 
\begin{itemize}
	\setlength\itemsep{.01em}
\item \lr{TP} : 
تعداد داده‌های حاوی خطا که به درستی تشخیص داده شدند
\item \lr{FP}: 
تعداد داده‌های سالم که به عنوان خطادار پیش‌بینی شدند
\item \lr{TN}:
تعداد داده‌های سالم که به درستی تشخیص داده شدند
\item \lr{FN}: 
تعداد داده‌های حاوی خطا که به عنوان داده‌ی سالم پیش‌بینی شدند

\end{itemize}


\begin{table}[H] 
		\renewcommand*{\arraystretch}{1.5}	
	\centering \caption{فرمول‌های محاسبه‌ی معیارهای ارزیابی}
	\label{tab:eval-metircs}
	\newcolumntype{C}{>{\centering\arraybackslash} m } 

	\begin{tabular}{|C{1.5cm} |C{2cm}|C{4.5cm}|C{6cm} |}
 
	\hline
	\hline
	نام معیار & نام لاتین & نحوه‌ی محاسبه & توضیح
		\\
	\hline
	\hline
	نرخ مثبت کاذب &
	\lr{False Positive Rate (PF)}  &
	$ \displaystyle \frac{FP}{TN+FP} $ &
	نسبت تعداد داده‌هایی که به اشتباه خطادار پیش‌بینی شده‌اند به تعداد کل داده‌های بدون خطا
	\\
	\hline
	صحت & 
		\lr{Accuracy} & $ \displaystyle \frac{TP+TN}{TP+FP+TN+FN}$ &
	نسبت	تعداد پیش‌بینی‌های درست به تعداد کل پیش‌بینی‌ها
		
	\\
	\hline
	دقت &
	\lr{Precision} & $\displaystyle \frac{TP}{TP+FP}$ &
نسبت تعداد داده‌هایی که به درستی خطادار پیش‌بینی شده‌اند به تعداد کل داده‌هایی که خطادار پیش‌بینی شده‌اند
	\\
	\hline
	بازخوانی & 
	\lr{Recall (PD)} & $\displaystyle \frac{TP}{TP+FN}$ &
	نسبت تعداد داده‌هایی که به درستی خطادار پیش‌بینی شده‌اند به تعداد کل داده‌های خطادار
	\\
	\hline
	معیار اف &
	\lr{F-Measure} & $ \displaystyle \frac{2 \times Precision \times Recall}{Precision + Recall}$ &
	از آنجا که در بین معیارهای دقت و بازخوانی مصالحه وجود دارد معیار اف ترکیبی از آن دو را در نظر می‌گیرد
	\\
	\hline
	\end{tabular}
\end{table}

دو معیار دیگر نیز که در پژوهش‌ها کاربرد دارند عبارتند از 
\lr{AUC }\LTRfootnote{Area under curve}  و 
\lr{AUCEC }\LTRfootnote{Area under cost-effectiveness curve }
که هر دو به مساحت زیر یک منحنی اشاره می‌کنند. \lr{AUC}  مساحت زیر نمودار
\lr{ROC }\LTRfootnote{Reciever operating characteristic}  
را اندازه‌گیری می‌کند. در نمودار \lr{ROC}،  محورهای عمودی و افقی را به ترتیب بازخوانی و  نرخ مثبت کاذب تشکیل می‌دهد.  با تغییر آستانه پیش‌بینی برای یک مدل می‌تواند میزان بازخوانی و  نرخ مثبت کاذب را تغییر داده و بدین ترتیب منحنی \lr{ROC} را رسم نمود. یک مدل بی نقص دارای مساحت زیر نمودار 1 است. برای یک مدل تصادفی  منحنی از مبدا به نقطه‌ی (1\lr{,}1) رسم خواهد شد. یک نمونه از منحنی \lr{ROC} در شکل \ref{fig:ROC} آمده است. \\

\begin{figure}
	\centering
	\includegraphics[width=.60\textwidth]{images/ROC.PNG}
	\caption{ نمونه‌ای از نمودار \lr{ROC} \cite{menzies2007data}}
	\label{fig:ROC}
\end{figure}

معیار \lr{AUCEC} معیاری است که تعداد خطوطی از برنامه که  توسط تیم تضمین کیفیت و یا توسعه دهندگان نیاز است بررسی و آزموده شود را در نظر می‌گیرد. ایده‌ی  موثر بودن از نظر هزینه\LTRfootnote{Cost-effectiveness}
برای مدل‌های ‌‌ خطا برای اولین بار توسط آریشلم و همکاران \cite{arisholm2007data} ارائه گردید. موثر بودن از نظر هزینه به این معنا است که چه تعداد خطا با بررسی و یا تست  \lr{$\%n$ } اول خطوط می‌توان یافت. به عبارت دیگر اگر یک مدل پیش‌بینی خطا بتواند تعداد خطای بیشتری را با بررسی و تلاش در آزمون کمتر، نسبت به باقی مدلها بیابد می‌توان گفت که تاثیر آن از نظر هزینه بیشتر است. دو منحنی در  قسمت راست شکل \ref{fig:AUCEC} برای دو مدل پیش‌بینی مختلف آمده است. هر دو مدل دارای سطح زیر نمودار یکسانی هستند اما زمانی که 20\lr{\%}  اول محور افقی در نظر گرفته می‌شود مدل 
\lr{P$_2$}
  کارایی بهتری دارد. نمودار سمت چپ مدلهای تصادفی، عملی\LTRfootnote {Practical} و بهینه را نشان می‌دهد.

\begin{figure}[H]
	\centering
	\includegraphics[width=.7\textwidth]{images/AUCEC.PNG}
	\begin{tabular}{c c c}
		\lr{O = optimal} & \lr{P = practical} &  \lr{R = random}\\
	 
	\end{tabular}
	\caption{ نمودار موثر بودن از نظر هزینه \cite{rahman2011bugcache}}
	\label{fig:AUCEC}
\end{figure}

معیارهایی که برای ارزیابی نتایج حاصل از روش رگرسیون به کار گرفته می‌شوند بر اساس همبستگی\LTRfootnote{Correlation} میان تعداد خطاهای پیش‌بینی شده و خطاهای واقعی محاسبه می‌شوند. نماینده‌ی این معیارها را می‌توان همبستگی اسپیرمن، پیرسون و $ R^2$ دانست \cite{nam2014survey}. 
\subsubsection{فرآیند پیش‌بینی خطا}
اکثریت پژوهش‌های پیش‌بینی خطا از روش‌های یادگیری ماشین  استفاده کرده‌اند. اولین گام در ساخت مدل پیش‌بینی تولید داده‌هایی با استفاده از آرشیو‌های نرم‌افزاری همچون سیستم‌های کنترل نسخه مانند گیت، سیستم‌های ردگیری مشکلات  مانند جیرا و آرشیو ایمیل‌ها است. هر یک از این داده‌ها بر اساس درشت دانگی پیش‌بینی می‌توانند نمایانگر یک سیستم، یک قطعه‌ی نرم‌افزاری\endnote{Component} (بسته\endnote{Package})، فایل کد منبع، کلاس و یا تابع باشد. یک داده حاوی چندین معیار (یا ویژگی) می‌باشد که از آرشیو‌های نرم‌افزاری استخراج شده و دارای برچسب "سالم" و "خطادار"  و یا تعداد خطاها است. پس از تولید داده‌ها با استفاده از معیارها و برچسب‌ها می‌توان به پیش پردازش داده‌ها پرداخت (مانند انتخاب معیار) که البته این امر اختیاری می‌باشد. پس از بدست آوردن مجموعه‌ی نهایی داده‌ها یک مدل پیش‌بینی را آموزش می‌دهیم که می‌تواند پیش‌بینی کند یک داده‌ی جدید حاوی خطا است یا خیر. تشخیص خطا خیز بودن داده\endnote{bug-proneness} معادل دسته بندی دودویی است و پیش‌بینی تعداد خطاها معادل رگرسیون می‌باشد. در شکل \ref{fig:prediction-process} فرآیند پیش‌بینی خطا نشان داده شده است. داده‌ها نمونه‌هایی هستند که می‌توانند خطادار  و بدون  خطا  بودن(   \lr{B = buggy} یا   \lr{C = clean} ) و یا تعداد خطا را نشان دهند. لازم به ذکر است که در یک مدل پیش‌بینی تنها از یک نوع از این داده‌ها استفاده می‌شود.

\begin{figure}[H]
	\centering
	\includegraphics[width=1.0\textwidth]{images/prediction-process.PNG}
	 \caption{فرآینده پیش‌بینی نرم‌افزار \cite{nam2014survey}}
	\label{fig:prediction-process}
\end{figure}
\subsubsection{معیارهای پیش بینی خطا}

معیارهای پیش بینی خطا نقش مهمی را در ساخت مدل پیش بینی ایفا می کنند. اکثریت معیارهای پیش بینی خطا را می توان به دو دسته ی تقسیم کرد: معیارهای کد و معیارهای فرآیند. معیارهای کد می توانند به طور مستقیم از کدهای منبع موجود جمع آوری شوند در حالی که معیارهای فرآیند  از اطلاعات تاریخی که در مخازن نرم افزاری مختلف آرشیو شده اند. نمونه از این مخازن نرم افزاری سیستم های کنترل نسخه و سیستم های ردگیری خطا است. معیار های فرآیند از نظر هزینه موثر تر از سایر معیارها هستند\cite{arisholm2010systematic}. در برخی از مقالات نیز معیارهای  پیش بینی خطا به سه دسته ی: معیارهای کد منبع سنتی، معیارهای شئ گرایی، معیارهای فرآیند تقسیم شده اند\cite{radjenovic2013software}.\\

معیارهای کد تحت عنوان معیارهای محصول 
\endnote{Product Metrics}
نیز شناخته می شوند و میزان پیچیدگی کد را می سنجند. فرض زمینه ای آنها این است که هرچه کد پیچیده تر باشد خطا خیز تر است. برای اندازه گیری پیچیدگی کد پژوهشگران معیار های مختلفی را ارائه داده اند که در ادامه به معرفی مهم ترین آنها پرداخته خواهد شد. معیارهای "اندازه" اندازه ی کلی و حجم کد را می سنجند. نماینده ی این معیارهای "تعداد خطوط" می باشد. اولین بار توسط آکیاما  \cite{akiyama1971example}  ارائه شد. هالستد  \cite{halstead1977elements} چندین معیار اندازه بر اساس  تعداد عملگرها و عملوند ها ارائه داده است. مک کیب معیارهای "حلقه ای" 
\endnote{Cyclomatic}
را پیشنهاد داد که این معیار با استفاده از تعداد گره ها، یالها و قطعات متصل در گراف جریان کنترلی کد منبع محاسبه می گردد\cite{mccabe1976complexity}. این معیارها نشان می دهند که راه های کنترلی به چه میزان پیچیده هستند. با ظهور زبانهای شئ گرایی و محبئبیت آنها معیارهای کد معیارهای کد برای این زبانها ارائه شد تا فرآیند توسعه بهبود یابد. نماینده ی معیارهای شی گرایی "چدامر و کمر" (\lr{CK}) می باشند\cite{chidamber1994metrics}. این معیارها طراحی شدند تبا توجه به خصیصه های زبانهای شئ گرایی مانند وارثت، زوجیت، همبستگی طراحی شده اند. بجز معیارهای \lr{CK} معیارهای شئ گرایی دیگری نیز بر اساس حجم و کمیت کد منبع پیشنهاد داده شده اند. همانند معیارهای "اندازه" معیارهای شئ گرایی تعداد نمونه های یک کلاس، توابع و غیره را می شمارند. \\
در ادامه به بررسی تعداد از معیارهای فرآیند پدراخته می شود که در این دسته شاخص محسوب می شوند. "ناگاپان و بال" 8 معیار تغییر نسبی کد را ارائه داده اند\cite{nagappan2005use}. به عنوان مثال یکی از معیار های آنها تعداد تجمعی خطوط اضافه و حذف شده بین دو نسخه از برنامه را می شمارد و بر تعداد خطوط برنامه تقسیم می کند. معیار دیگر تعداد فایلهای تغییر یافته از یک قطعه برنامه را بر تعداد فایلها تقسیم می کند. دسته ی دیگری از معیارهای فرآیند معیار تغییر هستند این معیار به عنوان مثال تعداد رفع خطاها، تعداد بازآرایی کد \endnote{Refactoring} و یا تعداد نویسنگان یک فایل را می شمارند. "موزر" و همکاران معیارهایی را ارائه داده اند که تعداد خطوط اضافه و کم شده را بدون در نظر گرفتن تعداد کل خطوط می شمارد. در عوض سن فایل ها و تعداد فایلهایی که کامیت می شوند در نظر گرفته می شود \cite{moser2008comparative}.  "بچلی" و همکاران معیارهای محبوبیت را بر اساس تحلیل رایانامه های آرشیو شده ی نویسنگان ارائه داده اند. ایده ی اصلی این معیارها این است که یک قطعه ی  نرم افزاری که در رایانامه ها درباره ی آن بیشتر صحبت شده است خطا خیز تر می باشد\cite{bacchelli2010popular}. "برد" و همکاران چهار معیار مالکیت بر اساس نویسندگان یک قطعه ارائه داده اند. مالکیت یک قطعه بر اساس نسبت کامیت هایی که افراد (مشارکت آنها) در یک قطعه  تعریف می شود. \\

"راجنویک" و همکاران در پژوهش خود به بررسی قاعده مند \endnote{Systematic Review} معیارهای پیش بینی خطا در مطالعات پیشین پرداخته اند.  طبق این پژوهش در 49\lr{\%} مطالعات از معیارهای شی گرایی، در 27 \lr{\%} معیارهای سنتی کد و در 26 \lr{\%} از معیارهای فرآیند استفاده شده است. با توجه به مطالعات بررسی شده دقت پیش بینی خطا تفاوت قابل توجهی با انتخاب معیارهای مختلف پیدا می کند. معیارهای شی گرایی و فرآیند موفق تر از معیارهای سنتی هستند. معیارهای سنتی پیش بینی قویا با معیارهای اندازه مانند تعداد خطوط کد همبستگی دارند و این دو توانایی پیش بینی خطا دارند اما جز بهترین معیارها نیستند. معیارهای شی گرایی بهتر از اندازه و پیچیدگی عمل می کنند و با این که با معیارهای اندازه همبستگی دارند اما ویژگی های بیشتری علاوه بر اندازه را دارند. معیارهای ایستای کد همانند اندازه، پیچیدگی و شئ گرایی به منظور بررسی یک نسخه از برنامه مفید هستند اما زمانی که با هر تکرار در فرآیند توسعه ی نرم افزار دقت پیش بینی آنها کاسته می شوند و معیارهای فرآیند در چنین شرایطی بهتر عمل می کنند. علارغم توانمندی بالقوه، معیارهای فرآیند در تعداد کمتری از پژوهش ها مورد استفاده قرار گرفته اند \cite{radjenovic2013software}. \\
 
آسترند و همکاران به بررسی این موضوع پرداختند که آیا اطلاعات درباره ی اینکه کدام توسعه دهنده یک فایل را اصلاح می کند قادر است که پیش بینی خطا را بهبود بخشد. در پژوهش قبلی آنها\cite{weyuker2008too} مشخص شده بود که تعداد کلی افراد توسعه دهنده در یک فایل می تواند در پیش بینی خطا تاثیر متوسطی داشته باشد. در این پژوهش تعدادی از متغیرهای کد منبع و فرآیند به همراه معیار مرتبط به توسعه دهنده در نظر گرفته شده است. تعداد خطاهایی که یک توسعه دهنده تولید می کند ثابت است و با سایر توسعه دهندگان فرق دارد. این تفاوت با در نظر گرفتن حجم کدی که یک توسعه دهنده اصلاح می کند مرتبط است و در نتیجه در نظر گرفتن یک نویسنده خاص نمی تواند به بهبود پیش بینی خطا کمک کند\cite{ostrand2010programmer}. \\

رحمان و دوانبو از جنبه های مختلف معیارهای فرآیند  را با سایر معیارها مقایسه کرده اند\cite{rahman2013and}. نتایج نشان می دهد  زمانی که مدل پیش بینی بر روی یک نسخه آموزش می بیند و در نسخه ی بعدی آزموده می شود معیارهای کد \lr{AUC} قابل قبولی دارند اما کمتر از معیارهای فرآیند و از نظر معیار 
\lr{AUCEC 20\%} 
بهتر از یک مدل تصادفی عمل نمی کند و این بدان معنی است که این معیارهعا از نظر هزینه چندان  موثر نیستند. همچنین معیارهای کد ایستا تر هستند، به این معنی که یعنی با تغییرات پروژه و تغییر در توزیع خطاها همچنان معیارها بدون تغییر باقی بمانند. معیار ایستا تمایل دارد یک فایل را انتشار\endnote{Release} پس از انتشار همچنان حاوی خطا معرفی کند. معیارهای ایستا به مدلهای راکد منجر می شوند که این مدلهابه سمت فایل های بزرگ با تراکم خطای کمتر بایاس دارند. به عنوان مثال حالتی را در نظر بگیرید که در یک پروژه فایلهای بزرگ و پیچیده ای وجود دارد که پس از چندین انتشار خطهای آنها برطرف می شود اما مدلهایی که بر اساس معیارهای کد ساخته شده اند همچنان این فایلها را به عنوان خطا خیز معرفی می کنند. از طرف دیگر حالتی را در نظر بگیرید که یک فایل با اندازه و پیچیدگی کم به تازگی به وجود آمده و یا تغییرات فراوان یافته است. مدلهای مبتنی بر کد به این فایلها توجه چندانی نخواهند کرد در حالیکه که این فایل ها مستعد وجود خطا هستند. بدین ترتیب معیارهای فرآیند بهتر از معیارهای کد عمل می کنند. 
 
 


\subsubsection{مدل‌های پیش‌بینی خطا}
اکثریت مدل‌های پیش‌بینی خطا بر اساس یادگیری ماشین می‌باشند. بر اساس اینکه چه چیزی پیش‌بینی شود (خطاخیز بودن یا تعداد خطا)، مدل‌ها به دو دسته‌ی کلی تقسیم می‌شوند، که عبارتند از دسته بندی و رگرسیون. با توسعه‌ی روش‌های جدیدتر یادگیری ماشین تکنیک‌های فعال و نیمه-نظارتی\LTRfootnote{Semi-Supervised} برای ساخت مدل‌های پیش‌بینی خطای کاراتر به کار گرفته شده است\cite{li2012sample}. علاوه بر مدل‌های یادگیری ماشین، مدل‌های غیر آماری مانند \lr{BugCache} پیشنهاد داده شده است \cite{kim2007predicting}. در میان روش‌های دسته بندی، 
\lr{Logestic Regression} ،
\lr{Naive Bayes} و
\lr{Decision Tree}
بیش از سایرین در پژوهش‌ها مورد استفاده قرار گرفته‌اند. همچنین در میان روش‌های رگرسیون 
\lr{Linear Regression} و 
\lr{Negetive Binomial Regression}  
به طور گسترده به کار گرفته شده‌اند \cite{nam2014survey}. \\
کیم\LTRfootnote{Kim} و همکاران \lr{BugCache} را ارائه داده‌اند که  اولویت موجودیت‌های خطاخیز در کش  را نگهداری  می‌کند. این روش از اطلاعات محلی خطاها مانند اطلاعات زمانی و مکانی بهره می‌گیرد. به عنوان مثال اگر خطا در یک موجودیت به تازگی به وجود آمده یا همراه با سایر موجودیت‌ها تغییر کرده است، آن موجودیت با احتمال بیشتری حاوی خطا خواهد بود.\\
اگرچه مدل‌های یادگیری مختلف می‌تواند  با توجه به داده‌های ورودی یکسان، متفاوت عمل کنند و کارایی یک روش نسبت به دیگری متفاوت باشد، با این حال پژوهشی که توسط آریشلم  و همکاران  \cite{arisholm2010systematic} انجام شده است نشان می‌دهد که تاثیر  تکنیک یادگیری در حد متوسطی است و کمتر از انتخاب معیار بر روی کارایی تاثیر گذار است.  \\

مالهوترا\LTRfootnote{Malhotra} با بکارگیری معیارهای سنتی کد، عملکرد تکنیک‌های یادگیری ماشین و رگرسیون را مقایسه کرده است\cite{malhotra2014comparative}. وی به منظور پیش پردازش نیز از آماره‌های توصیفی\LTRfootnote{Descriptive Statistics }  استفاده کرده است و داده‌های نامناسب را شناسایی نموده است. آماره‌های توصیفی می‌توانند شامل میانگین، کمینه، بیشینه، واریانس باشد. متغیرهای مستقلی که  واریانس کمی دارند ماژول‌ها را به خوبی متمایز نمی‌کنند و بعید است که مفید باشند و می‌توانند حذف شوند. یک روش رگرسیون و شش روش دسته بندی مورد آزمایش قرار گرفته‌اند که در میان آنها سه روش رایج و سه روش که کمتر مورد استفاده قرار می‌گیرند انتخاب شده‌اند. \lr{Logestic Regression} به عنوان روش رگرسیون انتخاب شده و نتایج نشان می‌دهد که روش‌های دسته بندی بهتر از روش رگرسیون عمل می‌کند. در میان روش‌های دسته بندی درخت تصمیم \LTRfootnote{Decision Tree} بهتر از سایرین عمل کرده است. 


\subsubsection{درشت‌دانگی پیش‌بینی }
در پژوهش‌های انجام شده مدل‌های پیش‌بینی در سطوح مختلفی از ریزدانگی ساخته شده‌اند از جمله: زیر سیستم، قطعه یا بسته، فایل یا کلاس، تابع و تغییر. هتا\LTRfootnote{Hata} و همکاران  پیش‌بینی در سطح تابع را ارائه داده‌اند و به این نتیجه رسیده‌اند که پیش‌بینی خطا در سطح تابع نسبت به سطوح درشت‌دانه‌تر از نظر هزینه موثرتر است \cite{hata2012bug}. کیم و همکاران نیز مدل جدیدی ارائه داده‌اند که "دسته بندی تغییر"\LTRfootnote{Change Classification} نام دارد. بر خلاف سایر مدل‌های پیش‌بینی، "دسته بندی تغییر می‌تواند به طور مستقیم به توسعه دهنده کمک کند. این مدل می‌تواند زمانی که توسعه دهنده تغییری در کد منبع ایجاد می‌کند و آنرا در سیستم کنترل نسخه ثبت می‌کند، نتایج آنی را فراهم کند.  از آنجا که این مدل بر اساس بیش از ده هزار ویژگی ساخته می‌شود، سنگین‌تر از آن است که در عمل مورد استفاده قرار گیرد\cite{kim2008classifying}. \\




\subsection{آزمون جهش و کاربردهای آن}
توسعه دهندگان و پژوهشگران حوزه ی نرم افزار علاقه مند به اندازه گیری موثر بودن مجموعه های آزمون می باشند. توسعه دهندگان به دنبال آن هستند که بدانند مجموعه آزمونهای آنها می تواند به خوبی خطاها را تشخیص دهد و پژوهشگران به دنبال مقایسه ی روشهای مختلف آزمون و اشکال زدایی\endnote{Debugging}  هستند. به طور ایده آل افراد تمایل دارند که بدانند تعداد خطاهایی که یک مجموعه آزمون می تواند شناسایی کند چه مقدار است اما از آنجا که خطاها نا شناخته هستند باید از اندازه گیری وکالتی \endnote{Proxy Measurment} استفاده شود. یکی از این اندازه گیری های شناخته شده امتیاز جهش \endnote{Mutation Score} می باشد که توانایی مجموعه آزمون در تمییز دادن نسخه ی اصلی برنامه از تعداد زیادی نسخه های متفاوت را اندازه گیری می کند. این نسخه های متفاوت که تنها یک تفاوت کوچک نحوی نسبت به برنامه ی اصلی دارند جهش یافته \endnote{Mutant} نامیده می شوند. امتیاز جهش درصد جهش یافته هایی  استکه توسط مجموعه آزمون از برنامه ی اصلی تمییز داده می شوند. بدین صورت که این جهش یافته های باعث شکست یک مورد آزمون می شود در حالی که در نسخه ی اصلی مجموعه ی آزمون با موفقیت اجرا می گردد. جهش یافته ها با تزریق خطاهای ساختگی به برنامه تحت آزمون  ساخته می شوند. این خطاهای ساختگی با استفاده از عملگرهای جهش که از پیش تعریف شده اند ساخته می شود. نمونه ی این عملگرها جایگزینی عملگرهای ریاضی یا رابطه ای، تغییر شرط شاخه \endnote{Branch Condition} و یا حذف یک عبارت است\cite{just2014mutants}. تحلیل آزمون در موارد زیر کاربرد دارد:
\begin{itemize}
	\setlength\itemsep{.01em}	
	\item 
	ارزیابی مجموعه آزمون
	\item 
	انتخاب مجموعه آزمون
	\item 
	 کمینه سازی مجموعه آزمون
	\item 
	 تولید مجموعه آزمون
	\item 
	مکان یابی خطا
	\item 
	پیش بینی خطا
\end{itemize}

روشهایی که از جهش یافته ها به منظور مکان یابی خطا استفاده می کنند دارای شباهت هایی با روشهای پیش بینی خطا هستند. در هر دوی این روشها از معیارهایی  کد منبع استفاده می شود تا احتمال وجود خطا محاسبه شود. دو تفاوت عمده ی این دو حوزه این است که اولا در مکان یابی خطا از روشهای یادگیری ماشین استفاده ی چندانی نمی شود . ثانیا در مکان یابی خطا وجود خطا به وسیله شکست مورد آزمون یا گزارش خطا محرز شده است. با توجه به شباهت های موجود میان این دو حوزه در ادامه چند مقاله که با استفاده از آزمون جهش به مکان یابی خطا پرداخته اند را بررسی می کنیم. \\

"مان" و همکاران در مقاله ی خود بر اساس دو فرض روشی به منظور مکان یابی خطا ارائه داده اند. فرض اول بیان می کند که  در یک برنامه ی حاوی خطا جهش و یا اصلاح یک عبارت خطا دار نسبت به جهش یک عبارت درست می تواند موارد آزمون بیشتری را  با موفقیت بگذراند فرض دوم  بیان می کند که جهش عبارات صحیح نسبت به جهش یک عبارت غلط موجب می شود موارد آزمون بیشتری شکست بخورند. بر اساس این دو فرض معیاری به نام "مشکوک بودن" \endnote{Suspiciousness} ارائه گردیده است که دو فرض را فرموله می کند. این معیار بر اساس تعداد موارد آزمون در نسخه ی اصلی و جهش یافته عمل می کند. سپس با رتبه بندی عبارات بر اساس این معیار عبارت حاوی خطا مشخص می گردد. در این پژوهش روش جدیدی نیز به منظور ارزیابی روش پیشنهادی ارائه شده است که برخی از مشکلات روش پیشین را بر طرف نموده است. در نهایت روش مکان یابی ارائه شده با دو روش ارزیابی شده و نتایج نشان می دهد فرضیات پژوهش درست بوده اند \cite{moon2014ask}. \\

"پاپاداکیس " و "تراوون" در مقاله ی خود به این نکته اشاره کرده اند استفاده از تحلیل جهش در گذشته به دلیل پر هزینه بودن چندان مورد توجه قرار نمی گرفته است اما امروزه با وجود ابزارهای مقیاس پذیر، نمونه گیری و انتخاب جهش می توان به خوبی از تحلیل جهش در انجام پژوهش های مختلف استفاده کرد\cite{papadakis2015metallaxis}. آنها روشی را برای مکان یابی خطا بر اساس دو مشاهده ارائه کرده اند. در مشاهده ی اول دیده می شود که خطای موجود در یک عبارت رفتار مشابهی با جهش در همان عبارت نشان می دهد. در مشاهده ی دیگر دیده می شود که اگر این دو خطا و جهش در دو عبارت متفاوت باشند رفتار متفاوتی خواهند داشت. منظور از رفتار مشابه موفقیت یا شکست در یک آزمون است. بر اساس این دو مشاهده معیاری برای مشکوک بودن عبارات تعیین می گردد. این پژوهش بیان می کند که مناسب بودن موارد آزمون تاثیر مستقیمی بر عملکرد روش مکان یابی خطا را دارد. همچینین یک مجموعه ی کوچک از جهش یافته ها می تواند به اندازه ی مجموعه ای کامل تا ثیر گذار باشد. \\

"جاست" و همکاران در پژوهش خود به بررسی این موضوع پرداخته اند که آیا جهش یافته ها می توانند جایگزین مناسبی برای خطاهای واقعی باشند یا خیر\cite{just2014mutants}. در پژوهش های گذشته بررسی شده بود که میان جهش یافته های ساده و پیچیده وابستگی وجود دارد ولی وابستگی میان جهش یافته های ساده و خطاهای واقعی مشخص نیست. جاست و همکاران دو مجموعه ی آزمون برای هر خطا در نظر گرفتند که مجموعه ی اول در نسخه ی حاوی خطا با موفقیت گذرانده می شود. مجموعه ی دوم در نسخه ی حاوی خطا شکست می خورد و در نسخه ی رفع خطا با موفقیت اجرا می شود. نتایج نشان می دهد که مجموعه ی آزمون دوم دارای امتیاز جهش بالاتری می باشد که نشان می دهد هر خطا به یک جهش یافته وابستگی دارد. لازم به ذکر است که سعی شده که دو مجموعه ی آزمون دارای پوشش یکسانی باشند زیرا پوشش بیشتر می تواند امتیاز جهش بیشتر بیانجامد. همچنین مشخص شد که  
73 \lr{\%} 
 خطاهای واقعی با جهش یافته هایی که  با عملگرهای متدوال تولید شده اند وابستگی دارند. در این پژوهش خطاهایی که با جهش یافته ها وابستگی ندارند در سه دسته قرار می گیرند : دسته اول نیازمند عملگرهای قوی تری هستند. دسته ی دوم نیازمند عملگرهای جدیدی هستند. دسته سوم با جهش یافته ها وابستگی ندارند.\\
 
 "هااو" و همکاران با ارایه ی مجموعه ای از معیارها و استفاده از یادگیری ماشین مدلی را ارائه داده اند که به وسیله ی آن بتوان تشخیص داد علت شکست در آزمون رگرسیون وجود خطا است یا منسوخ \endnote{Obsolete} شدن یک مورد آزمون. هفت معیار ارائه شده در این پژوهش مرتبط با گراف فراخوانی، تغییر در فایل ها و تعداد شکست در آزمونها بوده است.  هااو و همکاران به منظور به دست آوردن مجموعه داده ی حاوی خطا، به صورت دستی بر اساس استانداردهایی  از پیش تعریف شده خطاهایی را در کد قرار داده اند. بدین منظور عباراتی به صورت تصادفی که در سراسر کد محصول قرار دارند انتخاب شدند و به وسیله ی عملگرهای جهش خطاهایی تولید شده است. به منظور بدست آوردن آزمون های منسوخ شده، مجموعه آزمونهایی از نسخه ی قبلی برنامه بر روی کد محصولی نسخه ی بعدی بکار گرفته شده است. سپس با استفاده از روش ارزیابی میان دسته ای به آموزش و آزمایش مدل ساخته شده پرداخته می شود. نتایج پژوهش نشان می دهد که روش پیشنهادی زمانی که بر روی یک نسخه یا نسخه های مختلف از یک برنامه اعمال شود نتایج خوبی دارد (80\lr{\%} دقت) اما زمانی که بر روی برنامه های مختلف اعمال شود ( مجموعه آموزش از یک برنامه و آزمون بر روی برنامه ای دیگر) موثر نیست. نتایج نشان می دهد تکنیک ها مکان یابی خطا نتیجه ی مثبتی بر تشخیص نوع خطا که مربوط به محصول است یا آزمون، ندارد.\\
 
 بوئز و همکاران معیارهایی را مبتنی بر جهش معرفی کردند و مدلی را بر اساس این آنها ساخته اند\cite{bowes2016mutation} معیارهای سنتی و شئ گرایی نیز در ترکیب با آنها مورد استفاده قرار گرفته اند. 8 عملگر جهش در نظر گرفته شده و برای هر یک از آنها یک عملگر ایستا (بدون اجرای کد) و چهار معیار پویا ساخته شده و در مجموعه 40 معیار جهش ارائه شده است. از این جهت میان معیار ایستا و پویا تمایز قائل شده اند که اگر معیارهای ایستا به تنهایی  پیش بینی را بهبود بخشند از آنها بدون نیاز به موارد آزمون می توان استفاده کرد. نتایج پژوهش نشان می دهد که استفاده از معیارهای جهش بهبود قابل توجهی را در پیش بینی خطا به وجود می آورد. همچنین معیارهای پویا و ایستا در کنار یکدیگیر توانایی پیش بینی مناسبی دارند ولی استفاده ی جداگانه از آنها تاثیر چندان مثبتی نخواهد داشت. این پژوهش از دو جنبه حائز اهمیت می باشد. یکی اینکه اولین پژوهش در زمینه ی پیش بینی خطاست که از تحلیل جهش استفاده کرده است. دوم آنکه مشابه ترین پژوهش به پژوهش کنونی می باشد. 
 
 
 
 
 
 
 
 
 
 
 
 
 
 
 
 
 
 
 
 
 


\subsection{جمع بندی مطالعات پیشین}
هدف از پیش بینی خطا کمک به توسعه دهندگان نرم افزار و کاهش هزینه های نرم افزاری می باشد. روند پیش بینی خطا به این صورت است که با استفاده از مخازن نرم افزاری همانند سیستم کنترل نسخه و سیستم ردگیری خطا، اطلاعات کد منبع، خطا و اطلاعات تاریخی پروژه جمع آوری می شود. با توجه به معیارهای مختلف داده هایی استخراج می شود که هر داده دارای برچسب خطا دار یا حاوی خطا می باشد. قسمتی از این داده ها با استفاده از روشهای یادگیری ماشین مدلهای پیش بینی خطا را تولید می کنند و قسمت دیگر جهت آزمایش مدل به کار گرفته می شود.\\

معیارهای متداول در پیش بینی \lr{Precision} و \lr{Recall} می باشند. این معیارها دارای نواقصی هستند. به عنوان مثال مدلی که همه ی داده ها را خطا دار معرفی می کند دارای \lr{Recall}=1 است و مسلما این مدل کارایی مناسبی ندارد. معیار \lr{F-Measure}  میانگین هارمونیک دو معیار قبلی است و نواقص آنها را بر طرف می کند. یکی از معیار های رایج برای مقایسه ی مدلهای یادگیری ماشین \lr{AUC} می باشد. هرچه این مساحت بیشتر باشد و زودتر محور عمودی آن به یک برسد مدل کارایی بهتری دارد. با استفاده از معیار \lr{َAUCEC} می توان موثر بودن مدل از نظر هزینه را سنجید. معمولا چند درصد اول از منحی مربوطه در نظر گرفته می شود و مساحت آن محاسبه می شود. \\

معیارهای مورد استفاده را می توان به سه دسته ی معیار سنتی کد، معیار شئ گرایی و معیار فرآیند تقسیم کرد. در برخی از منابع نیز دو دسته ی کلی معیار کد و معیار فرآیند تقسیم شده اند. معیارهای اندازه جزء معیارهای ابتدایی و موثر هستند و معیارهای پیچیدگی و شئ گرایی همبستگی فراوانی با معیارهای اندازه دارند. با این حال معیارهای شئ گرایی دارای وابستگی فراوانی با معیار های اندازه هستند با این جال معیارهای شئ گرایی دارای توانایی بیشتری هستند. معیارهای فرآیند از جنبه های مختلفی  مانند عدم رکود در تکرار های چرخه ی تولید نرم افزارم و موثر بودن از نظر هزینه از سایر معیارهای برتری دارد. علارغم توانمندی بالقوی معیارهای فرآیند در پیش بینی خطا، این معیارها در پژوهش های کمتری مورد تحقیق قرار گرفته اند. \\

در پژوهش های مختلف از روشهای یادگیری ماشین متفاوتی استفاده شده است. در صورتی که هدف پیش بینی تعداد خطاها باشد از رگرسیون و در صورتی که هدف پیش بینی حاوی خطا بودن باشد از دسته بندی استفاده می شود. پژوهش \cite{arisholm2010systematic}  نشان داده است که روش دسته بندی تاثیر متوسطی بر کارایی پیش بینی خطا دارد و انتخاب معیار مهم تر است. \\

جهش یافته به نسخه از برنامه گفته می شود که در یک عبارت از آن با استفاده از عملگرهایی از پیش تعریف شده ی جهش، یک تغییر نحوی ایجاد می شود. به درصدی از جهش یافته ها که موجب شکست در مجموعه آزمون می گردد امتیاز جهش گفته می شود. در ابتدا از امتیاز جهش برای میزان موثر بودن مجموعه آزمون استفاده می شد و سپس کاربردهای دیگری همچون انتخاب، رتبه بندی و کمینه کردن مجموعه آزمون پیدا کرده است. همچنین در پژوهش های اخیر جهت مکان یابی خطا و  و پیش بینی خطا مورد استفاده قرار گرفته است. در پژوهش \cite{just2014mutants} نشان داده شده است که جهش یافته هایی  که با عملگرهای جهش ساده تولید شده اند می توانند تا 73 \lr{\%} خطاهای واقعی را شبیه سازی کنند و ازین جهت جایگزین مناسبی برای خطاهای واقعی باشند. 


\begin{table}[H] 
	\centering \caption{جدول مشخصات پژوهشهای مرور شده در حوزه‌ی پيش بيني خطا}
	\label{tab:survey}
	
	\begin{tabular}{|c|c|c|c|c|c|c|}
		\hline
		\hline
		مقاله & معیار   & تکنیک یادگیری &  ریزدانگی &روش ارزیابی & نوع پروژه‌ها&‌ زبان پروژه‌ها   \\
		\hline
		\hline
		
		
	
\end{tabular}
\end{table}












 
 
 



\section{راهکار پیشنهادی}
\label{sec:method}
با  مطالعات مروری انجام شده نقاطی از این حوزه که نیازمند پژوهش بیشتر هستند تا بتوان به وسیله‌ی آن به ارائه‌ی روشی کاراتر در پیش‌بینی خطا پرداخت مشخص شد. مقاله‌ی \cite{bowes2016mutation} اولین مقاله‌ای است که  یک  روش پیش‌بینی خطا با استفاده از تحلیل جهش ارائه نموده  است و این موضوع نیازمند تحقیق بیشتر است. از طرف دیگر بر طبق مقاله‌ی \cite{radjenovic2013software} استفاده از معیارهای فرآیند علی‌رغم توانایی بالقوه‌ای که در پیش‌بینی خطا دارند، در پژوهش‌های کمتری مورد بررسی قرار گرفته‌اند. یکی از دلایل آن می‌تواند نو ظهور بودن این معیارها نسبت به سایرین باشد. معیارهای فرآیند از جنبه‌های مختلف نیز از سایر معیار‌ها برتری دارند \cite{rahman2013and}. با توجه به شواهد مطرح شده، این پژوهش قصد دارد به بررسی تاثیر ترکیب معیارهای مبتنی بر جهش با معیارهای فرآیند و همچنین ارائه‌ی معیارهای جدیدی که بر پایه‌ی تحلیل جهش و تاریخچه‌ی نرم‌افزار باشد بپردازد.\\
سوالاتی که این پژوهش قصد پاسخ به آنها را دارد عبارتند از:

\begin{enumerate}
	\item  
	به کارگیری معیارهای فرآیند به همراه معیارهای جهش به چه میزان به بهبود پیش‌بینی خطا کمک می‌کند؟
	\item 
	معیارهای جهش ارائه شده مبتنی بر تاریخچه‌ی نرم‌افزار به چه میزان موجب بهبود پیش‌بینی خطا می‌شود؟
	\item 
	وزن دهی به معیارهای فرآیند با توجه به جهش آیا می‌تواند به بهبود پیش‌بینی خطا بیانجامد؟  
\end{enumerate}

پرسش اول با توجه به مقاله‌ی \cite{bowes2016mutation} مطرح شده که در آن بررسی به کارگیری معیارهای جهش و فرآیند را در پژوهش‌های آتی توصیه می‌کند.  همچنین  معیار جهش یک معیار  مرتبط با کد است. مقاله‌ی \cite{rahman2013and}  بیان می‌کند که معیارهای کد ایستا هستند و تمایل دارند که یک موجودیت را در انتشارهای متوالی حاوی خطا معرفی کنند. حال شرایطی را در نظر بگیرید که که امتیاز جهش در یک موجودیت کم باشد و دلیل آن کافی نبودن مجموعه آزمون باشد چراکه توسعه‌دهندگان از درست بودن کد اطمینان دارند یا اینکه پس از انتشارهای متوالی خطاها بر طرف شده است. چنین موجودیتی حاوی خطا نیست اما با توجه به معیار جهش خطا‌خیز است. با در نظر گرفتن معیارهای فرآیند در مورد این موجودیت که نشان می‌دهند پایدار و بدون تغییر است از میزان خطا‌خیز بودن آن کاسته می‌شود و انتظار می‌رود کارایی مدل پیش‌بینی بهبود یابد. 
برای پاسخ به این پرسش مجموعه معیارهای جهش  از پژوهش \cite{bowes2016mutation}  و معیارهای فرآیند از پژوهش \cite{rahman2013and} انتخاب می‌شوند. در جداول  \ref{tab:process-metircs} و \ref{tab:mutation-metircs} معیارهای مورد نظر آورده شده است.   با استفاده از مجموعه داده‌ای  که فراهم خواهد شد دو مدل پیش‌بینی ساخته می‌شود. یکی با استفاده از معیارهای فرآیند به تنهایی و دیگری با استفاده از معیارهای جهش در کنار معیارهای فرآیند. سپس  با استفاده از روش‌های ارزیابی مطرح شده در قسمت \ref{subsec:eval}  دو مدل مقایسه می‌شوند. \\
\begin{table}[H] 
	\renewcommand*{\arraystretch}{1}	
	\centering \caption{معیارهای فرآیند 
	\cite{rahman2013and}}
	\label{tab:process-metircs}
	\newcolumntype{C}{>{\centering\arraybackslash} m } 
	
	\begin{tabular}{|c|c|}
		
		\hline
		\hline
		نام معیار  & توضیح
		\\
		\hline
		\hline
		\lr{COMM } & تعداد ثبت در سیستم کنترل نسخه
		\\
		\hline
		\lr{ADEV} & تعداد توسعه‌دهندگان 
		فعال
		\\ 
		\hline
		\lr{DDEV} & تعداد توسعه‌دهندگان 
		متفاوت
		\\ 
		\hline
		\lr{ADD} &  تعداد نرمال‌سازی شده‌ی خطوط اضافه شده
		\\ 
		\hline
		\lr{DEL}  & تعداد نرمال‌سازی شده‌ی خطوط حذف شده
		\\ 
		\hline
		\lr{OWN} &  تعداد خطوطی که مالک فایل مشارکت کرده
		\\ 
		\hline
		\lr{MINOR} & تعداد خطوط مشارکت‌کنندگان جزئی
		\\ 
		\hline
	\end{tabular}
\end{table}

\begin{table}[H] 
	\renewcommand*{\arraystretch}{1}	
	\centering \caption{معیارهای جهش 
		\cite{bowes2016mutation}}
	\label{tab:mutation-metircs}
	\newcolumntype{C}{>{\centering\arraybackslash} m } 
	
	\begin{tabular}{|c|c|}
		
		\hline
		\hline
		نام معیار &  توضیح
		\\
		\hline
		\hline
		\lr{MuNOM } &   تعداد جهش‌یافته‌های تولید شده
		\\
		\hline
		\lr{MuNOC} &   تعداد جهش‌یافته‌های پوشش‌داده شده توسط آزمون‌ها
		\\
		\hline
		\lr{MuNNC} &   تعداد جهش‌یافته‌های پوشش‌داده نشده توسط آزمون‌ها
		\\
		\hline
		\lr{MuNMS} &   امتیاز جهش‌یافته‌های تولید شده
		\\
		\hline
		\lr{MuNMSC} &   امتیاز جهش‌یافته‌های پوشش‌داده شده توسط آزمون‌ها
		\\
		\hline
		
	\end{tabular}
\end{table}

در پاسخ به سوال دوم چهار معیار زیر مطرح می‌شود:
\begin{enumerate}

	\item  
	\textbf{
	تعداد جهش‌یافته‌های تولید شده‌ی جدید نسبت به نسخه‌ی قبلی برنامه: }همانطور که در مقاله‌ی \cite{just2014mutants} مطرح شده جهش‌یافته‌ها جایگزین خوبی برای خطاهای واقعی می‌باشند. زمانی که تعداد جهش‌یافته‌های جدید زیاد باشد یعنی تغییراتی که خطا‌خیز‌ترهستند بیشتر است. 
	\item 
	\textbf{
	تعداد جهش‌یافته‌های متمایز در تمام نسخه‌های قبلی:} این معیار نشان می‌دهد موجودیت مورد بررسی به چه میزان سابقه‌ی تغییراتی را دارد که احتمال بروز خطا را افزایش می‌دهد.
	
	\item 
	\textbf{
	میزان تغییرات مثبت امتیاز جهش دو به دو در تمام نسخه‌های قبلی:}
تغییرات امتیاز جهش نشان از تغییرات در برنامه و آزمون‌های نرم‌افزار است.     این معیار نشان می‌دهد این تغییرات به چه میزان در جهت بهبود کیفیت نرم‌افزار بوده. چراکه امتیاز بالاتر جهش نشان از کیفیت بهتر آزمون‌ها و در نتیجه نرم‌افزار است. 
	\item 
	\textbf{
	میزان تغییرات منفی امتیاز جهش در تمام نسخه‌های قبلی:}
این معیار مشابه معیار سوم عمل می‌کند با این تفاوت که میزان تغییرات در خلاف جهت بهبود نرم‌افزار را می‌سنجد. 	
\end{enumerate}
با استفاده از مجموعه داده‌ی فراهم شده دو مدل پیش‌بینی ساخته می‌شود. یکی با استفاده از معیارهای فرآیند به تنهایی و دیگری با استفاده از معیارهای مطرح شده. سپس با استفاده از روش‌های ارزیابی مطرح شده در قسمت \ref{subsec:eval}  دو مدل مقایسه می‌شوند.\\

پرسش سوم با توجه به مطالب گفته شده در مقاله‌ی \cite{rahman2013and} مطرح شده که بیان می‌کند معیارها هر چقدر هم که پویا باشند (دچار رکود نشوند، مانند معیارهای فرآیند) زمانی در پیش‌بینی خطا مفید هستند که همراه با ایجاد خطا باشند.  دو معیار مطرح در میان معیارهای فرآیند تعداد خطوط اضافه شده و حذف شده است. با توجه به تعداد جهش‌یافته‌هایی که  اضافه  و یا حذف هر خط ایجاد می‌کند، اضافه یا کم شدن خطوط وزن دهی می‌شود. دو معیار جدید به صورت زیر ساخته می‌شود و تاثیر آنها بر کارایی مشابه پرسش اول و دوم سنجیده می‌شود. \\
\begin{latin}
	
	$M_1 =\ number\ of\ lines\ added\ \times \ number\ of\ muatants\ derived$\\
	
	$M_2 =\ number\ of\ lines\ deleted\ \times \ number\ of\ mutants\ derived$\\
\end{latin}



 
\section{جمع بندی و کارهای آتی}
\label{sec:future}
 در این گزارش مقالات مختلف در حوزه‌ی پیش بینی خطا و استفاده از آزمون جهش بررسی شد. همچنین نقاطی که  کاستی دارد و یا نیاز به پژوهش بیشتر است شناسایی شدند. نکات مفید در ارائه‌ی معیارهای پیش بینی معرفی شدند و با توجه به آنها معیارهایی ارائه شد. این معیارها مبتنی بر جهش و تاریخچه‌ی نرم افزار هستند.\\
 
   در آینده مدلهای پیش بینی با استفاده از این معیارها ساخته خواهد شد و با توجه به روشهای گفته شده ارزیابی خواهند شد.  به منظور افزایش کارایی، در صورت نیاز در معیارهای ارائه شده یا نحوه‌ی ساخت مدل    بازنگری خواهد شد و در نهایت نگارش پایانامه انجام خواهد گرفت. در جدول \ref{tab:schedule} زمانبندی انجام پروژه آمده است.

\begin{table}[H] 
	\centering \caption{زمانبندی انجام پروژه}
	\label{tab:schedule}

	\begin{tabular}{|c|c|c|c|c|}
		\hline
		\hline
		 ردیف & عنوان & مدت زمان لازم & پیشرفت & زمان شروع
	 \\
		\hline
		\hline
1& مطالعه‌ی روش‌های پیش بینی خطا & یک ماه& 100\% & مرداد 96
		\\
		\hline
2 &مطالعه‌ی کاربردهای آزمون جهش  &دو ماه &100\%  &مهر 96 
		\\
		\hline
3 & جمع آوری داده‌های پروژه‌های نرم افزاری &  یک ماه& 10\% & آذر 96
		\\
		\hline
4 & استخراج معیارهای انتخاب شده از داده‌ها &  یک ماه& 0\% & فروردین 97
\\
\hline
5 & ساخت مدل پیش بینی &  یک ماه& 0\% & اردیبهشت 97
\\
\hline
6 & ارزیابی مدل ارائه شده و بهبود کاستی‌ها &  یک ماه& 0\% & خرداد 97
\\
\hline
7 & نگارش پایانامه &  یک ماه& 0\% & تیر 97\\


\hline
		
	 
	\end{tabular}
\end{table}

 
%

%\listoffigures
%\listoftables

% insert each of your chapters with a \inlcude{filename} command as below
%\doublespacing %yek safheye khali fek konam ijad mikoneh
%\onehalfspacing
%\linespread{1.5}
%----------------------------------------------------------------------
%\singlespacing
\linespread{1}
\small
\setlength{\parskip}{0pt}
\setlength{\parsep}{0pt}



\setlength{\headsep}{0in}
\addtolength{\topmargin}{-1.3cm}
\addtolength{\textheight}{1cm}
\addtolength{\oddsidemargin}{0.5cm}
%\addtolength{\evensidemargin}{-0.6cm}
%\addtolength{\textwidth}{2.1cm}

 

%\renewcommand{}
 \renewcommand{\bibname}{ \rightline{\rl{مراجع}}}

\begin{latin}
	 
	\baselineskip=.8\baselineskip
%	\bibliographystyle{./styles/packages/unsrtabbrv}
	\bibliographystyle{IEEEtran} % such as plain
	% Uncomment next line to include uncited references
	% \nocite{*}
	
	 \bibliography{References}
	
\end{latin}

%
%% If do not have appendix then comment following 3 lines
%\doublespacing
%\appendix
%%\include{peyvast1}
%%\include{peyvast2}d
%% -------------------------------------------------------
%\pagenumbering{gobble}
\fancyhf{} % sets both header and footer to nothing
\renewcommand{\headrulewidth}{0pt}


\newpage 
\  
\newline
%\newline


\renewcommand{\notesname}{ \rightline{\rl{واژه نامه}}}


\begin{latin}
	
	\renewcommand{\enotesize}{\normalsize}
	\renewcommand\enoteformat{%
		\raggedright
		
		\makebox[0pt][r]{\theenmark \ - \rule{0pt}{\dimexpr\ht\strutbox}}%
	}

	\begin{multicols}{2}

	
	\begingroup
	
	
	\parindent 0pt
	\parskip .1em
	\def\enotesize{\normalsize}
	\theendnotes
	\endgroup
\end{multicols}
\end{latin}
 


\end{document}
