\subsection{جمع بندی مطالعات پیشین}
هدف از پیش بینی خطا کمک به توسعه دهندگان نرم افزار و کاهش هزینه های نرم افزاری می باشد. روند پیش بینی خطا به این صورت است که با استفاده از مخازن نرم افزاری همانند سیستم کنترل نسخه و سیستم ردگیری خطا، اطلاعات کد منبع، خطا و اطلاعات تاریخی پروژه جمع آوری می شود. با توجه به معیارهای مختلف داده هایی استخراج می شود که هر داده دارای برچسب خطا دار یا حاوی خطا می باشد. قسمتی از این داده ها با استفاده از روشهای یادگیری ماشین مدلهای پیش بینی خطا را تولید می کنند و قسمت دیگر جهت آزمایش مدل به کار گرفته می شود.\\

معیارهای متداول در پیش بینی \lr{Precision} و \lr{Recall} می باشند. این معیارها دارای نواقصی هستند. به عنوان مثال مدلی که همه ی داده ها را خطا دار معرفی می کند دارای \lr{Recall}=1 است و مسلما این مدل کارایی مناسبی ندارد. معیار \lr{F-Measure}  میانگین هارمونیک دو معیار قبلی است و نواقص آنها را بر طرف می کند. یکی از معیار های رایج برای مقایسه ی مدلهای یادگیری ماشین \lr{AUC} می باشد. هرچه این مساحت بیشتر باشد و زودتر محور عمودی آن به یک برسد مدل کارایی بهتری دارد. با استفاده از معیار \lr{َAUCEC} می توان موثر بودن مدل از نظر هزینه را سنجید. معمولا چند درصد اول از منحی مربوطه در نظر گرفته می شود و مساحت آن محاسبه می شود. \\

معیارهای مورد استفاده را می توان به سه دسته ی معیار سنتی کد، معیار شئ گرایی و معیار فرآیند تقسیم کرد. در برخی از منابع نیز دو دسته ی کلی معیار کد و معیار فرآیند تقسیم شده اند. معیارهای اندازه جزء معیارهای ابتدایی و موثر هستند و معیارهای پیچیدگی و شئ گرایی همبستگی فراوانی با معیارهای اندازه دارند. با این حال معیارهای شئ گرایی دارای وابستگی فراوانی با معیار های اندازه هستند با این جال معیارهای شئ گرایی دارای توانایی بیشتری هستند. معیارهای فرآیند از جنبه های مختلفی  مانند عدم رکود در تکرار های چرخه ی تولید نرم افزارم و موثر بودن از نظر هزینه از سایر معیارهای برتری دارد. علارغم توانمندی بالقوی معیارهای فرآیند در پیش بینی خطا، این معیارها در پژوهش های کمتری مورد تحقیق قرار گرفته اند. \\

در پژوهش های مختلف از روشهای یادگیری ماشین متفاوتی استفاده شده است. در صورتی که هدف پیش بینی تعداد خطاها باشد از رگرسیون و در صورتی که هدف پیش بینی حاوی خطا بودن باشد از دسته بندی استفاده می شود. پژوهش \cite{arisholm2010systematic}  نشان داده است که روش دسته بندی تاثیر متوسطی بر کارایی پیش بینی خطا دارد و انتخاب معیار مهم تر است. \\

جهش یافته به نسخه از برنامه گفته می شود که در یک عبارت از آن با استفاده از عملگرهایی از پیش تعریف شده ی جهش، یک تغییر نحوی ایجاد می شود. به درصدی از جهش یافته ها که موجب شکست در مجموعه آزمون می گردد امتیاز جهش گفته می شود. در ابتدا از امتیاز جهش برای میزان موثر بودن مجموعه آزمون استفاده می شد و سپس کاربردهای دیگری همچون انتخاب، رتبه بندی و کمینه کردن مجموعه آزمون پیدا کرده است. همچنین در پژوهش های اخیر جهت مکان یابی خطا و  و پیش بینی خطا مورد استفاده قرار گرفته است. در پژوهش \cite{just2014mutants} نشان داده شده است که جهش یافته هایی  که با عملگرهای جهش ساده تولید شده اند می توانند تا 73 \lr{\%} خطاهای واقعی را شبیه سازی کنند و ازین جهت جایگزین مناسبی برای خطاهای واقعی باشند. 











