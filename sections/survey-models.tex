
\subsubsection{مدل‌های پیش‌بینی خطا}
اکثریت مدل‌های پیش‌بینی خطا بر اساس یادگیری ماشین می‌باشند. بر اساس اینکه چه چیزی پیش‌بینی شود (خطاخیز بودن یا تعداد خطا)، مدلها به دو دسته‌ی کلی تقسیم می‌شوند، که عبارتند از دسته بندی و رگرسیون. با توسعه‌ی روشهای جدیدتر یادگیری ماشین تکنیک‌های فعال و نیمه-نظارتی\endnote{Semi-Supervised} برای ساخت مدلهای پیش‌بینی خطای کاراتر به کار گرفته شده است\cite{li2012sample}. علاوه بر مدلهای یادگیری ماشین، مدلهای غیر آماری مانند \lr{BugCache} پیشنهاد داده شده است \cite{kim2007predicting}. در میان روشهای دسته بندی، 
\lr{Logestic Regression} ،
\lr{Naive Bayes} و
\lr{Decision Tree}
بیش از سایرین در پژوهش‌ها مورد استفاده قرار گرفته‌اند. همچنین در میان روشهای رگرسیون 
\lr{Linear Regression} و 
\lr{Negetive Binomial Regression}  
به طور گسترده به کار گرفته شده‌اند \cite{nam2014survey}. \\
کیم\LTRfootnote{Kim} و همکاران \lr{BugCache} را ارائه داده‌اند که  اولویت موجودیت‌های خطاخیز در کش  را نگهداری  می‌کند. این روش از اطلاعات محلی خطاها مانند اطلاعات زمانی و مکانی بهره می‌گیرد. به عنوان مثال اگر خطا در یک موجودیت به تازگی به وجود آمده یا همراه با سایر موجودیت‌ها تغییر کرده است، آن موجودیت با احتمال بیشتری حاوی خطا خواهد بود.\\
اگرچه مدلهای یادگیری مختلف می‌تواند  با توجه به داده‌های ورودی یکسان، متفاوت عمل کنند و کارایی یک روش نسبت به دیگری متفاوت باشد، با این حال پژوهشی که توسط آریشلم  و همکاران  \cite{arisholm2010systematic} انجام شده است نشان می‌دهد که تاثیر  تکنیک یادگیری در حد متوسطی است و کمتر از انتخاب معیار بر روی کارایی تاثیر گذار است.  \\

مالهوترا\LTRfootnote{Malhotra} با بکارگیری معیارهای سنتی کد به مقایسه‌ی عملکرد تکنیک‌های یادگیری ماشین و رگرسیون پرداخته است\cite{malhotra2014comparative}. وی به منظور پیش پردازش نیز از آماره‌های توصیفی\endnote{Descriptive Statistics }  استفاده کرده است و داده‌های نامناسب را شناسایی نموده است. آماره‌های توصیفی می‌توانند شامل میانگین، کمینه، بیشینه، واریانس باشد. متغیرهای مستقلی که  واریانس کمی دارند ماژول‌ها را به خوبی متمایز نمی‌کنند و بعید است که مفید باشند و می‌توانند حذف شوند. یک روش رگرسیون و شش روش دسته بندی مورد آزمایش قرار گرفته‌اند که در میان آنها سه روش رایج و سه روش که کمتر مورد استفاده قرار می‌گیرند انتخاب شده‌اند. \lr{Logestic Regression} به عنوان روش رگرسیون انتخاب شده و نتایج نشان می‌دهد که روش‌های دسته بندی بهتر از روش رگرسیون عمل می‌کند. در میان روشهای دسته بندی درخت تصمیم \endnote{Decision Tree} بهتر از سایرین عمل کرده است. 


\subsubsection{درشت‌دانگی پیش‌بینی }
در پژوهش‌های انجام شده مدلهای پیش‌بینی در سطوح مختلفی از ریزدانگی ساخته شده‌اند از جمله: زیر سیستم، قطعه یا بسته، فایل یا کلاس، تابع و تغییر. هتا\LTRfootnote{Hata} و همکاران  پیش‌بینی در سطح تابع را ارائه داده‌اند و به این نتیجه رسیده‌اند که پیش‌بینی خطا در سطح تابع نسبت به سطوح درشت‌دانه‌تر از نظر هزینه موثرتر است \cite{hata2012bug}. کیم و همکاران نیز مدل جدیدی ارائه داده‌اند که "دسته بندی تغییر"\endnote{Change Classification} نام دارد. بر خلاف سایر مدلهای پیش‌بینی، "دسته بندی تغییر می‌تواند به طور مستقیم به توسعه دهنده کمک کند. این مدل می‌تواند زمانی که توسعه دهنده تغییری در کد منبع ایجاد می‌کند و آنرا در سیستم کنترل نسخه ثبت می‌کند، نتایج آنی را فراهم کند.  از آنجا که این مدل بر اساس بیش از ده هزار ویژگی ساخته می‌شود، سنگین‌تر از آن است که در عمل مورد استفاده قرار گیرد\cite{kim2008classifying}. \\


