\subsubsection{فرآیند پیش بینی خطا}
اکثریت پژوهش های پیش بینی خطا از روش های یادگیری ماشین  استفاده کرده اند. اولین گام در ساخت مدل پیش بینی تولید داده هایی با استفاده از آرشیو های نرم افزاری همچون سیستم های کنترل نسخه مانند گیت، سیستم های ردگیری مشکلات \endnote{Issue Tracking Systems} مانند جیرا،  آرشیو ایمیل ها و غیره است. هر یک از این داده ها بر اساس درشت دانگی پیش بینی می توانند نمایانگر یک سیستم، یک قطعه ی نرم افزاری\endnote{Component} (بسته\endnote{Package})، فایل کد منبع، کلاس و یا تابع باشد. یک داده حاوی چندین معیار (یا ویژگی) می باشد که از آرشیو های نرم افزاری استخراج شده و دارای برچسب "سالم" و "خطادار"  و یا تعداد خطاها می باشد. پس از تولید داده ها با استفاده از معیارها و برچسب ها می توان به پیش پردازش داده ها پرداخت (مانند انتخاب معیار) که البته این امر اختیاری می باشد. پس از بدست آوردن مجموعه ی نهایی داده ها یک مدل پیش بینی را آموزش می دهیم که می تواند پیش بینی کند یک داده ی جدید حاوی خطا است یا خیر. تشخیص خطا خیز بودن داده\endnote{bug-proneness} معادل دسته بندی دودویی است و پیش بینی تعداد خطاها معادل رگرسیون می باشد. 