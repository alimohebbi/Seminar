\section{راهکار پیشنهادی}
\label{sec:method}
با  مطالعات مروری انجام شده نقاطی از این حوزه که نیازمند پژوهش بیشتر هستند تا بتوان به وسیله‌ی آن به ارائه‌ی روشی کاراتر در پیش بینی خطا پرداخت مشخص شد. مقاله‌ی \cite{bowes2016mutation} اولین مقاله‌ای است که  به ارائه‌ی روش پیش بینی خطا با استفاده از تحلیل جهش پرداخته است و این موضوع نیازمند تحقیق بیشتر است. از طرف دیگر بر طبق مقاله‌ی \cite{radjenovic2013software} استفاده از معیارهای فرآیند علارغم توانایی بالقوه‌ای که در پیش بینی خطا دارند، در پژوهش‌های کمتری مورد بررسی قرار گرفته‌اند. یکی از دلایل آن می‌تواند نو ظهور بودن این معیارها نسبت به سایرین باشد. معیارهای فرآیند از جنبه‌های مختلف نیز از سایر معیار‌ها برتری دارند \cite{rahman2013and}. با توجه به شواهد مطرح شده، این پژوهش قصد دارد به بررسی تاثیر ترکیب معیارهای مبتنی بر جهش با معیارهای فرآیند و همچنین ارائه‌ی معیارهای جدیدی که بر پایه‌ی تحلیل جهش و تاریخچه‌ی نرم افزار باشد بپردازد.\\
سوالاتی که این پژوهش قصد پاسخ به آنها را دارد عبارتند از:

\begin{enumerate}
	\item  
	به کارگیری معیارهای فرآیند به همراه معیارهای جهش به چه میزان به بهبود پیش بینی خطا کمک می‌کند؟
	\item 
	معیارهای جهش ارائه شده مبتنی بر تاریخچه‌ی نرم افزار به چه میزان موجب بهبود پیش بینی خطا می‌شود؟
	\item 
	وزن دهی به معیارهای فرآیند با توجه به جهش آیا می‌تواند به بهبود پیش بینی خطا بیانجامد؟  
\end{enumerate}

پرسش اول با توجه به مقاله‌ی \cite{bowes2016mutation} مطرح شده که در آن بررسی به کارگیری معیارهای جهش و فرآیند را در پژوهش‌های آتی توصیه می‌کند.  همچنین  معیار جهش یک معیار  مرتبط با کد است. مقاله‌ی \cite{rahman2013and}  بیان می‌کند که معیارهای کد ایستا هستند و تمایل دارند که یک موجودیت را انتشار پس از انتشار حاوی خطا معرفی کنند. حال شرایطی را در نظر بگیرید که که امتیاز جهش در یک موجودیت کم باشد و دلیل آن کافی نبودن مجموعه آزمون باشد چراکه توسعه دهندگان از درست بودن کد اطمینان دارند یا اینکه پس از انتشارهای متوالی خطاها بر طرف شده است. چنین موجودیتی حاوی خطا نیست اما با توجه به معیار جهش خطا خیز است. با در نظر گرفتن معیارهای فرآیند در مورد این موجودیت که نشان می‌دهند پایدار و بدون تغییر است از میزان خطا خیز بودن آن کاسته می‌شود و انتظار می‌رود کارایی مدل پیش بینی بهبود یابد. 
برای پاسخ به این پرسش مجموعه معیارهای جهش  از پژوهش \cite{bowes2016mutation}  و معیارهای فرآیند از پژوهش \cite{rahman2013and} انتخاب می‌شوند. با استفاده از مجموعه داده‌ای  که فراهم خواهد شد دو مدل پیش بینی ساخته می‌شود. یکی با استفاده از معیارهای فرآیند به تنهایی و دیگری با استفاده از معیارهای جهش در کنار معیارهای فرآیند. سپس  با استفاده از روشهای ارزیابی مطرح شده در قسمت \ref{subsec:eval}  به مقایسه‌ی دو مدل پرداخته می‌شود. \\

در پاسخ به سوال دوم چهار معیار زیر مطرح می‌شود:
\begin{enumerate}

	\item  
	\textbf{
	تعداد جهش یافته‌های تولید شده‌ی جدید نسبت به نسخه‌ی قبلی برنامه: }همانطور که در مقاله‌ی \cite{just2014mutants} مطرح شده جهش یافته‌ها جایگزین خوبی برای خطاهای واقعی می‌باشند. زمانی که تعداد جهش یافته‌های جدید زیاد باشد یعنی تغییراتی که خطا خیز تر هستند بیشتر است. 
	\item 
	\textbf{
	تعداد جهش یافته‌های متمایز در تمام نسخه‌های قبلی:} این معیار نشان می‌دهد موجودیت مورد بررسی به چه میزان سابقه‌ی تغییراتی را درد که احتمال بروز خطا را افزایش می‌دهد.
	
	\item 
	\textbf{
	میزان تغییرات مثبت امتیاز جهش دو به دو در تمام نسخه‌های قبلی:}
تغییرات امتیاز جهش نشان از تغییرات در برنامه و آزمونهای نرم افزار است.     این معیار نشان می‌دهد این تغییرات به چه میزان در جهت بهبود کیفیت نرم افزار بوده. چراکه امتیاز بالاتر جهش نشان از کیفیت بهتر آزمونها و در نتیجه نرم افزار است. 
	\item 
	\textbf{
	میزان تغییرات منفی امتیاز جهش در تمام نسخه‌های قبلی:}
این معیار مشابه معیار سوم عمل می‌کند با این تفاوت که میزان تغییرات در خلاف جهت بهبود نرم افزار را می‌سنجد. 	
\end{enumerate}
با استفاده از مجموعه داده‌ی فراهم شده دو مدل پیش بینی ساخته می‌شود. یکی با استفاده از معیارهای فرآیند به تنهایی و دیگری با استفاده از معیارهای مطرح شده. سپس با استفاده از روشهای ارزیابی مطرح شده در قسمت \ref{subsec:eval}  به مقایسه‌ی دو مدل پرداخته می‌شود.\\

پرسش سوم با توجه به مطالب گفته شده در مقاله‌ی \cite{rahman2013and} مطرح شده که بیان می‌کند معیارها هر چقدر هم که پویا باشند (دچار رکود نشوند، مانند معیارهای فرآیند) زمانی در پیش بینی خطا مفید هستند که همراه با ایجاد خطا باشند.  دو معیار مطرح در میان معیارهای فرآیند تعداد خطوط اضافه شده و حذف شده است. با توجه به تعداد جهش یافته‌هایی که  اضافه  و یا حذف هر خط ایجاد می‌کند، اضافه یا کم شدن خطوط وزن دهی می‌شود. دو معیار جدید به صورت زیر ساخته می‌شود و تاثیر آنها بر کارایی مشابه پرسش اول و دوم سنجیده می‌شود. \\
\begin{latin}
	
	$M_1 =\ number\ of\ lines\ added\ \times \ number\ of\ muatants\ derived$\\
	
	$M_2 =\ number\ of\ lines\ deleted\ \times \ number\ of\ mutants\ derived$\\
\end{latin}



 