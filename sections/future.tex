\section{جمع بندی و کارهای آتی}
\label{sec:future}
 در این گزارش مقالات مختلف در حوزه‌ی پیش بینی خطا و استفاده از آزمون جهش بررسی شد. همچنین نقاطی که  کاستی دارد و یا نیاز به پژوهش بیشتر است شناسایی شدند. نکات مفید در ارائه‌ی معیارهای پیش بینی معرفی شدند و با توجه به آنها معیارهایی ارائه شد. این معیارها مبتنی بر جهش و تاریخچه‌ی نرم افزار هستند.\\
 
   در آینده مدلهای پیش بینی با استفاده از این معیارها ساخته خواهد شد و با توجه به روشهای گفته شده ارزیابی خواهند شد.  به منظور افزایش کارایی، در صورت نیاز در معیارهای ارائه شده یا نحوه‌ی ساخت مدل    بازنگری خواهد شد و در نهایت نگارش پایانامه انجام خواهد گرفت. در جدول \ref{tab:schedule} زمانبندی انجام پروژه آمده است.

\begin{table}[H] 
	\centering \caption{زمانبندی انجام پروژه}
	\label{tab:schedule}

	\begin{tabular}{|c|c|c|c|c|}
		\hline
		\hline
		 ردیف & عنوان & مدت زمان لازم & پیشرفت & زمان شروع
	 \\
		\hline
		\hline
1& مطالعه‌ی روش‌های پیش بینی خطا & یک ماه& 100\% & مرداد 96
		\\
		\hline
2 &مطالعه‌ی کاربردهای آزمون جهش  &دو ماه &100\%  &مهر 96 
		\\
		\hline
3 & جمع آوری داده‌های پروژه‌های نرم افزاری &  یک ماه& 10\% & آذر 96
		\\
		\hline
4 & استخراج معیارهای انتخاب شده از داده‌ها &  یک ماه& 0\% & فروردین 97
\\
\hline
5 & ساخت مدل پیش بینی &  یک ماه& 0\% & اردیبهشت 97
\\
\hline
6 & ارزیابی مدل ارائه شده و بهبود کاستی‌ها &  یک ماه& 0\% & خرداد 97
\\
\hline
7 & نگارش پایانامه &  یک ماه& 0\% & تیر 97\\


\hline
		
	 
	\end{tabular}
\end{table}
