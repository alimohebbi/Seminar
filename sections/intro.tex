%\chapter{راهنمای استفاده از کلاس}
\thispagestyle{empty}
\noindent
\renewcommand*\thesection{\arabic{section}}
\textbf{\large{چکیده:}}
تکنیک‌های پیش‌بینی خطا می‌تواند تیم‌های تضمین کیفیت را یاری دهد تا تلاش خود را  روی قسمت‌های حاوی خطا متمرکز کنند. با استفاده از ابزارهای آماری مدرن  ساخت و به کارگیری مدل‌های پیش‌بینی سریع و آسان شده است. پژوهشگران سعی دارند تا با بهبود قسمت‌های مختلف پیش‌بینی خطا مدل‌های کاراتری را تولید کنند. تعدادی از این تلاش‌ها عبارتند از: ‌تولید مجموعه داده‌های دقیقتر، بکارگیری روش‌های نوین یادگیری ماشین و ارائه‌ی معیارهای جدید. معیارهای پیش‌بینی از اصلی ترین اجزای مدل پیش‌بینی  خطااست که بیشترین تاثیر را بر نتیجه دارد.  گزارش با‌ ارائه‌ی مدل مبتنی بر جهش و فرآیند سعی در ارتقای کارایی پیش‌بینی خطا دارد. 
\\\\
\textbf{واژه‌های کلیدی:}\textit{ 
پیش‌بینی خطا، معیارهای نرم افزار، آزمون نرم افزار، جهش
}\\\\
\section{مقدمه}
\label{sec:intro}
  توسعه‌دهندگان نرم‌افزار از طریق گزارش خطا در سیستم‌های ردگیری مشکلات\endnote{Issue Tracking Systems} و یا شکست در آزمون نرم‌افزار متوجه حضور خطا می‌شوند و پس از آن به جستجوی محل خطا و درک مشکل  نرم‌افزار می‌پردازند. کشف زود هنگام خطاها موجب صرفه‌جویی در زمان و هزینه می‌شود و فرآیند اشکال زدایی را تسهیل می‌بخشد. روش‌های پیش‌بینی خطا با به کارگیری تکنیک‌های یادگیری ماشین مانند دسته ‌بندهای \lr{Navie Bayes } و \lr{Logistic Regression} ، مدلی مناسب ارائه می‌دهند و وجود خطا را در یک موجودیت که می‌تواند فایل و یا تابع باشد پیش‌بینی می‌کنند. داده‌های مورد نیاز مانند کد منبع و گزارش خطا، به منظور آموزش مدل از سیستم‌های ردگیری مشکلات و مخازن منابع کد\endnote{Source Code Repositories} استخراج می‌شوند. \\
  
 به منظور ساخت یک مدل یادگیری ماشین لازم است که از داده‌های موجود ویژگی\endnote{Feature} و یا صفاتی\endnote{Attributes}  برای هر موجودیت استخراج شوند. این ویژگی‌ها به همراه برچسب سالم و یا خطادار بودن موجودیت جهت آموزش مدل یادگیری بکار می‌رود.  ویژگی  در پژوهش‌های حوزه‌ی پیش‌بینی خطا معیار\endnote{Metric}  نامیده می‌شود.  معیارها را می‌توان به دو دسته‌ی کلی کد منبع و فرآیند تقسیم کرد. در پژوهش‌های قبلی نشان داد شده که در بسیاری از موارد معیارهای مبتنی بر فرآیند بهتر از سایر معیارها عمل می‌کنند\cite{rahman2013and}\cite{radjenovic2013software}. \\
 
  یکی از مشکلات این روش‌ها عدم دقت کافی در پیش‌بینی خطا به علت محدودیت‌های عملکردی مدل‌های یادگیری می‌باشد. پژوهش‌گران برای بهبود پیش‌بینی خطا سعی کرده‌اند که معیارهای جدید ارائه دهند تا داده‌های بیشتر و موثرتری فراهم شود. در برخی پژوهش‌ها از مدل بدست آمده در پروژه‌های دیگر در پیش‌بینی خطای یک پروژه استفاده شده تا کمبود داده‌های آموزش جبران شود و یا تکنیک‌های یادگیری ماشین به‌روزتری مانند یادگیری گروهی\endnote{Ensmble Learning} به کار گرفته شده است. در پژوهش‌های اخیر  با استفاده از موارد آزمون و مفاهیم آزمون جهش معیارهایی مانند تعداد جهش‌یافته‌هایی که توسط آزمون‌ها پوشش داده شده مطرح  شده است و در ساخت مدل پیش‌بینی خطا \cite{bowes2016mutation} و مکان‌یابی خطا  \cite {papadakis2015metallaxis}استفاده شده است. در این پژوهش‌ها معیارهای مطرح شده تنها در یک مقطع زمانی در نظر گرفته شده و مدل پیش‌بینی با ترکیب این معیارها با معیارهای کد منبع ساخته شده است.\\
  
 در این پژوهش معیارهای جهش ارائه شده در پژوهش‌های قبلی در کنار معیارهای فرآیند قرار می‌گیرند و تاثیر ترکیب این دو نوع  معیار در کارایی مدل پیش‌بینی حاصل ارزیابی می‌شود. همچنین معیارهای جهش جدیدی با در نظر گرفتن تاریخچه‌ی نرم افزار ارائه می‌گردد و یک مدل پیش‌بینی با به کارگیری معیارهای فرآیند و معیارهای ارائه شده ساخته می‌شود و میزان بهبود پیش‌بینی خطا ارزیابی می‌شود.  
