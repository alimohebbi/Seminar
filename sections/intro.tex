%\chapter{راهنمای استفاده از کلاس}
\thispagestyle{empty}
\noindent
\renewcommand*\thesection{\arabic{section}}
\textbf{\large{چکیده:}}
در متن  چکيده از ذكر مقدمات و كليات خودداري شود و مستقيماً به مسأله مورد مطالعه و اهداف آن، اساس كار، و ميزان موفقيت اين مطالعه با استناد به نتايج كار به طور مختصر اشاره شود. در چكيده از ذكر جزييات كار، شكل‌ها، جدول ها، فرمول‌ها، و مراجع‌ پرهيز كنيد. چکيده حداكثر شامل 150 كلمه مي باشد.\\\\
\textbf{واژه‌های کلیدی:}\textit{ حداكثر ٨ واژه، مجزا شده با ويرگول}\\\\
\section{مقدمه}
\label{sec:intro}
در این قسمت باید فضای کلی مسئله تعریف گردد که شامل موارد ذیل می‌باشد:\\
- بیان تعاریف و مفاهیم پایه‌ای حوزه مورد بررسی پژوهش.\\
- تعريف رسمی مسئله مطرح و مورد بررسی در پژوهش.\\
- بیان اهمیت و فایده موضوع با ذکر کاربردهای آن.\\
- بیان چالش‌های مطرح در حوزه مورد بررسی پژوهش.\\
 
