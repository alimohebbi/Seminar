\section{بررسی کارهای مرتبط پیشین}
\label{sec:survey}

\subsection{پیش بینی خطا}
\subsubsection{فرآیند پیش بینی خطا}
اکثریت پژوهش های پیش بینی خطا از روش های یادگیری ماشین  استفاده کرده اند. اولین گام در ساخت مدل پیش بینی تولید داده هایی با استفاده از آرشیو های نرم افزاری همچون سیستم های کنترل نسخه مانند گیت، سیستم های ردگیری مشکلات \endnote{Issue Tracking Systems} مانند جیرا،  آرشیو ایمیل ها و غیره است. هر یک از این داده ها بر اساس درشت دانگی پیش بینی می توانند نمایانگر یک سیستم، یک قطعه ی نرم افزاری\endnote{Component} (بسته\endnote{Package})، فایل کد منبع، کلاس و یا تابع باشد. یک داده حاوی چندین معیار (یا ویژگی) می باشد که از آرشیو های نرم افزاری استخراج شده و دارای برچسب "سالم" و "خطادار"  و یا تعداد خطاها می باشد. پس از تولید داده ها با استفاده از معیارها و برچسب ها می توان به پیش پردازش داده ها پرداخت (مانند انتخاب معیار) که البته این امر اختیاری می باشد. پس از بدست آوردن مجموعه ی نهایی داده ها یک مدل پیش بینی را آموزش می دهیم که می تواند پیش بینی کند یک داده ی جدید حاوی خطا است یا خیر. تشخیص خطا خیز بودن داده\endnote{bug-proneness} معادل دسته بندی دودویی است و پیش بینی تعداد خطاها معادل رگرسیون می باشد. 
\subsubsection{معیارهای ارزیابی}
معیارهای ارزیابی را می توان به دسته ی کلی  معیارهای دسته بندی و رگرسیون تقسیم کرد.  معیارهای دسته بندی را می توان با استفاده از ماتریس درهم ریختگی\endnote{Confusion Matrix} محاسبه نمود. در ماتریس درهم ریختگی پیش بینی خطا، عناصر را به صورت زیر تعریف می شوند.  همچنین نحوه ی محاسبه ی معیارها در جدول \ref{tab:eval-metircs} آمده است. 
\begin{itemize}
\item \lr{TP} : 
داده های حاوی خطا که به درستی تشخیص داده شدند
\item \lr{FP}: 
  داده های سالم که به عنوان خطادار پیش بینی شدند
\item \lr{TN}:
داده های سالم که به درستی تشخیص داده شدند
\item \lr{FN}: 
داده های حاوی خطا که به عنوان داده ی سالم پیش بینی شدند

\end{itemize}


\begin{table}[H] 
		\renewcommand*{\arraystretch}{1.5}	
	\centering \caption{فرمول های محاسبه ی معیارهای ارزیابی}
	\label{tab:eval-metircs}
	\begin{tabular}{|c |c|}
	\hline
	\hline
	نام معیار & نحوه ی محاسبه
		\\
	\hline
	\hline
	\lr{False Positive Rate (PF)}  &
	$  \frac{FP}{TN+FP} $
	\\
	\hline
		\lr{Accuracy} & $ \frac{TP+TN}{TP+FP+TN+FN}$
	\\
	\hline
	\lr{Precision (PD)} & $\frac{TP}{TP+FP}$
	\\
	\hline
	\lr{Recall} & $\frac{TP}{TP+FN}$
	\\
	\hline
	\lr{F-Measure} & $ \frac{2 \times Precision \times Recall}{Precision + Recall}$
	\\
	\hline
	\end{tabular}
\end{table}

دو معیار دیگر نیز که در پژوهش ها کاربرد دارد عبارتند از 
\lr{AUC } \endnote{Area under curve}  و 
\lr{AUCEC } \endnote{Area under cost-effectiveness curve }
که هر دو به مساحت زیر یک منحنی اشاره می کنند. \lr{AUC}  مساحت زیر نمودار
\lr{ROC } \endnote{Reciever operating characteristic}  
را اندازه گیری می کند. در نمودار \lr{ROC}،  محورهای عمودی و افقی را به ترتیب \lr{PD } و  \lr{PF} تشکیل می دهد.  با تغییر آستانه پیش بینی برای یک مدل می تواند میزان \lr{PF } و \lr{PD } را تغییر داده و بدین ترتیب منحنی \lr{AUC} را رسم نمود. یک مدل بی نقص دارای مساحت زیر نمودار 1 است. برای یک مدل تصادفی  منحنی از مبدا به نقطه ی(1و1) رسم خواهد شد. 




