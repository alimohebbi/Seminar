\section{بررسی کارهای مرتبط پیشین}
\label{sec:survey}

\subsection{پیش بینی خطا}
\subsubsection{فرآیند پیش بینی خطا}
اکثریت پژوهش های پیش بینی خطا از روش های یادگیری ماشین  استفاده کرده اند. اولین گام در ساخت مدل پیش بینی تولید داده هایی با استفاده از آرشیو های نرم افزاری همچون سیستم های کنترل نسخه مانند گیت، سیستم های ردگیری مشکلات \endnote{Issue Tracking Systems} مانند جیرا،  آرشیو ایمیل ها و غیره است. هر یک از این داده ها بر اساس درشت دانگی پیش بینی می توانند نمایانگر یک سیستم، یک قطعه ی نرم افزاری\endnote{Component} (بسته\endnote{Package})، فایل کد منبع، کلاس و یا تابع باشد. یک داده حاوی چندین معیار (یا ویژگی) می باشد که از آرشیو های نرم افزاری استخراج شده و دارای برچسب "سالم" و "خطادار"  و یا تعداد خطاها می باشد. پس از تولید داده ها با استفاده از معیارها و برچسب ها می توان به پیش پردازش داده ها پرداخت (مانند انتخاب معیار) که البته این امر اختیاری می باشد. پس از بدست آوردن مجموعه ی نهایی داده ها یک مدل پیش بینی را آموزش می دهیم که می تواند پیش بینی کند یک داده ی جدید حاوی خطا است یا خیر. تشخیص خطا خیز بودن داده\endnote{bug-proneness} معادل دسته بندی دودویی است و پیش بینی تعداد خطاها معادل رگرسیون می باشد. 
\subsubsection{معیارهای ارزیابی}
معیارهای ارزیابی را می توان به دسته ی کلی  معیارهای دسته بندی و رگرسیون تقسیم کرد.  معیارهای دسته بندی را می توان با استفاده از ماتریس درهم ریختگی\endnote{Confusion Matrix} محاسبه نمود. در ماتریس درهم ریختگی پیش بینی خطا، عناصر را به صورت زیر تعریف می شوند.  همچنین نحوه ی محاسبه ی معیارها در جدول \ref{tab:eval-metircs} آمده است. 
\begin{itemize}
	\setlength\itemsep{.01em}
\item \lr{TP} : 
داده های حاوی خطا که به درستی تشخیص داده شدند
\item \lr{FP}: 
  داده های سالم که به عنوان خطادار پیش بینی شدند
\item \lr{TN}:
داده های سالم که به درستی تشخیص داده شدند
\item \lr{FN}: 
داده های حاوی خطا که به عنوان داده ی سالم پیش بینی شدند

\end{itemize}


\begin{table}[H] 
		\renewcommand*{\arraystretch}{1.5}	
	\centering \caption{فرمول های محاسبه ی معیارهای ارزیابی}
	\label{tab:eval-metircs}
	\begin{tabular}{|c |c|}
	\hline
	\hline
	نام معیار & نحوه ی محاسبه
		\\
	\hline
	\hline
	\lr{False Positive Rate (PF)}  &
	$  \frac{FP}{TN+FP} $
	\\
	\hline
		\lr{Accuracy} & $ \frac{TP+TN}{TP+FP+TN+FN}$
	\\
	\hline
	\lr{Precision (PD)} & $\frac{TP}{TP+FP}$
	\\
	\hline
	\lr{Recall} & $\frac{TP}{TP+FN}$
	\\
	\hline
	\lr{F-Measure} & $ \frac{2 \times Precision \times Recall}{Precision + Recall}$
	\\
	\hline
	\end{tabular}
\end{table}

دو معیار دیگر نیز که در پژوهش ها کاربرد دارد عبارتند از 
\lr{AUC } \endnote{Area under curve}  و 
\lr{AUCEC } \endnote{Area under cost-effectiveness curve }
که هر دو به مساحت زیر یک منحنی اشاره می کنند. \lr{AUC}  مساحت زیر نمودار
\lr{ROC } \endnote{Reciever operating characteristic}  
را اندازه گیری می کند. در نمودار \lr{ROC}،  محورهای عمودی و افقی را به ترتیب \lr{PD } و  \lr{PF} تشکیل می دهد.  با تغییر آستانه پیش بینی برای یک مدل می تواند میزان \lr{PF } و \lr{PD } را تغییر داده و بدین ترتیب منحنی \lr{AUC} را رسم نمود. یک مدل بی نقص دارای مساحت زیر نمودار 1 است. برای یک مدل تصادفی  منحنی از مبدا به نقطه ی(1و1) رسم خواهد شد. \\
معیار \lr{AUCEC} معیاری است که تعداد خطوطی از برنامه که  توسط تیم تضمین کیفیت و یا توسعه دهنهده گان نیاز است بررسی و آزموده شود را در نظر می گیرد. ایده ی به موثر بودن از نظر هزینه 
\endnote{Cost-effectiveness}
برای مدلهای پیش بینی خطا برای اولین بار توسط آریشلم و همکاران \cite{arisholm2007data} ارائه گردید. موثر بودن از نظر هزینه به این معنا است که چه تعداد خطا با بررسی و یا تست  \lr{$\%n$ } اول خطوط می توان یافت. به عبارت دیگر اگر یک مدل پیش بینی خطا بتواند تعداد خطای بیشتری را با بررسی و تلاش در آزمون کمتر نسبت به باقی مدلها بیابد می توان گفت که تاثیر آن از نظر هزینه بیشتر است. 

\subsubsection{معیارهای پیش بینی خطا}

معیارهای پیش بینی خطا نقش مهمی را در ساخت مدل پیش بینی ایفا می کنند. اکثریت معیارهای پیش بینی خطا را می توان به دو دسته ی تقسیم کرد: معیارهای کد و معیارهای فرآیند. معیارهای کد می توانند به طور مستقیم از کدهای منبع موجود جمع آوری شوند در حالی که معیارهای فرآیند  از اطلاعات تاریخی که در مخازن نرم افزاری مختلف آرشیو شده اند. نمونه از این مخازن نرم افزاری سیستم های کنترل نسخه و سیستم های ردگیری خطا است. \\
معیارهای کد تحت عنوان معیارهای محصول 
\endnote{Product Metrics}
نیز شناخته می شوند و میزان پیچیدگی کد را می سنجند. فرض زمینه ای آنها این است که هرچه کد پیچیده تر باشد خطا خیز تر است. برای اندازه گیری پیچیدگی کد پژوهشگران معیار های مختلفی را ارائه داده اند که در ادامه به معرفی مهم ترین آنها پرداخته خواهد شد. معیارهای "اندازه" اندازه ی کلی و حجم کد را می سنجند. نماینده ی این معیارهای "تعداد خطوط" می باشد. اولین بار توسط آکیاما  \cite{akiyama1971example}  ارائه شد. هالستد  \cite{halstead1977elements} چندین معیار اندازه بر اساس  تعداد عملگرها و عملوند ها ارائه داده است. مک کیب معیارهای "حلقه ای" 
\endnote{Cyclomatic}
را پیشنهاد داد که این معیار با استفاده از تعداد گره ها، یالها و قطعات متصل در گراف جریان کنترلی کد منبع محاسبه می گردد\cite{mccabe1976complexity}. این معیارها نشان می دهند که راه های کنترلی به چه میزان پیچیده هستند. با ظهور زبانهای شئ گرایی و محبئبیت آنها معیارهای کد معیارهای کد برای این زبانها ارائه شد تا فرآیند توسعه بهبود یابد. نماینده ی معیارهای شی گرایی "چدامر و کمر" (CK) می باشند\cite{chidamber1994metrics}. این معیارها طراحی شدند تبا توجه به خصیصه های زبانهای شئ گرایی مانند وارثت، زوجیت، همبستگی طراحی شده اند. بجز معیارهای CK معیارهای شئ گرایی دیگری نیز بر اساس حجم و کمیت کد منبع پیشنهاد داده شده اند. همانند معیارهای "اندازه" معیارهای شئ گرایی تعداد نمونه های یک کلاس، توابع و غیره را می شمارند. \\
در ادامه به بررسی تعداد از معیارهای فرآیند پدرداخته می شود که در این دسته شاخص محسوب می شوند. "ناگاپان و بال" 8 معیار تغییر نسبی کد را ارائه داده اند\cite{nagappan2005use}. به عنوان مثال یکی از معیار های آنها تعداد تجمعی خطوط اضافه و حذف شده بین دو نسخه از برنامه را می شمارد و بر تعداد خطوط برنامه تقسیم می کند. معیار دیگر تعداد فایلهای تغییر یافته از یک قطعه برنامه را بر تعداد فایلها تقسیم می کند. دسته ی دیگری از معیارهای فرآیند معیار تغییر هستند این معیار به عنوان مثال تعداد رفع خطاها، تعداد بازآرایی کد \endnote{Refactoring} و یا تعداد نویسنگان یک فایل را می شمارند. "موزر" و همکاران معیارهایی را ارائه داده اند که تعداد خطوط اضافه و کم شده را بدون در نظر گرفتن تعداد کل خطوط می شمارد. در عوض سن فایل ها و تعداد فایلهایی که کامیت می شوند در نظر گرفته می شود \cite{moser2008comparative}.  "بچلی" و همکاران معیارهای محبوبیت را بر اساس تحلیل رایانامه های آرشیو شده ی نویسنگان ارائه داده اند. ایده ی اصلی این معیارها این است که یک قطعه ی  نرم افزاری که در رایانامه ها درباره ی آن بیشتر صحبت شده است خطا خیز تر می باشد\cite{bacchelli2010popular}. "برد" و همکاران چهار معیار مالکیت بر اساس نویسندگان یک قطعه ارائه داده اند. مالکیت یک قطعه بر اساس نسبت کامیت هایی که افراد (مشارکت آنها) در یک قطعه  تعریف می شود. 
\subsubsection{مدل های پیش بینی خطا}
اکثریت مدلهای پیش بینی خطا بر اساس یادگیری ماشین می باشند. بر اساس اینکه چه چیزی پیش بینی شود (خطاخیز بودن یا تعداد خطا)، مدلها به دو دسته ی کلی تقسیم می شوند، دسته بندی و رگرسیون. با توسعه ی روشهای جدیدتر یادگیری ماشین تکنیک های فعال و نیمه-نظارتی \endnote{Semi-Supervised} برای ساخت مدلهای پیش بینی خطای بهتر بکار گرفته شده است\cite{li2012sample}. علاوه بر مدلهای یادگیری ماشین مدلهای غیر آماری مانند \lr{BugCache} پیشنهاد داده شده است \cite{kim2007predicting}. در میان روشهای دسته بندی، 
\lr{Logestic Regression} ،
\lr{Naive Bayes} و
\lr{Decision Tree}
بیش از سایرین در پژوهش ها مورد استفاده قرار گرفته اند. همچنین در میان روشهای رگرسیون 
\lr{Linear Regression} و 
\lr{Negetive Binomial Regression}  
به طور گسترده بکار گرفته شده اند \cite{nam2014survey}. \\
کیم و همکاران \lr{BugCache} را ارائه داده اند که از اولویت موجودیت های خطا خیز در کش نگهداری می کند. این روش از اطلاعات محلی خطاها مانند اطلاعات زمانی و مکانی بهره می گیرد. به عنوان مثال اگر خطا در یک موجودیت به تازگی به وجود آمده یا همراه با سایر موجودیت ها تغییر کرده است، آن موجودیت با احتمال بیشتری حاوی خطا خواهد بود. 
\subsubsection{درشت دانگی پیش بینی }
در پژوهش های انجام شده مدلهای پیش بینی در سطوح مختلفی از ریز دانگی ساخته شده اند از جمله: زیر سیستم، قطعه یا بسته، فایل یا کلاس، تابع و تغییر. "هتا" و همکاران در پیش بینی در سطح تابع را ارائه داده اند و به این نتیجه رسیده اند که پیش بینی خطا در سطح تابع نسبت به سطوح درشت دانه تر از نظر هزینه موثر تر است \cite{hata2012bug}. کیم و همکاران نیز مدل جدیدی ارائه داده اند که "دسته بندی تغییر" \endnote{Change Classification} نام دارد. بر خلاف سایر مدلهای پیش بینی، "دسته بندی تغییر می تواند به طور مستقیم به توسعه دهنده کمک کند. این مدل می تواند زمانی که توسعه دهنده تغییر در کد منبع ایجاد می کند و آنرا در سیستم کنترل نسخه کامیت می کند، نتایج آنی را فراهم کند.  از آنجا که این مدل بر اساس بیش از ده هزار ویژگی ساخته می شود، سنگین تر از آن است که در عمل مورداستفاده قرار گیرد\cite{kim2008classifying}. 

\subsection{آزمون جهش و کاربردهای آن}
توسعه دهندگان و پژوهشگران حوزه ی نرم افزار علاقه مند به اندازه گیری موثر بودن مجموعه های آزمون می باشند. توسعه دهندگان به دنبال آن هستند که بدانند مجموعه آزمونهای آنها می تواند به خوبی خطاها را تشخیص دهد و پژوهشگران به دنبال مقایسه ی روشهای مختلف آزمون و اشکال زدایی\endnote{Debugging}  هستند. به طور ایده آل افراد تمایل دارند که بدانند تعداد خطاهایی که یک مجموعه آزمون می تواند شناسایی کند چه مقدار است اما از آنجا که خطاها نا شناخته هستند باید از اندازه گیری وکالتی \endnote{Proxy Measurment} استفاده شود. یکی از این اندازه گیری های شناخته شده امتیاز جهش \endnote{Mutation Score} می باشد که توانایی مجموعه آزمون در تمییز دادن نسخه ی اصلی برنامه از تعداد زیادی نسخه های متفاوت را اندازه گیری می کند. این نسخه های متفاوت که تنها یک تفاوت کوچک نحوی نسبت به برنامه ی اصلی دارند جهش یافته \endnote{Mutant} نامیده می شوند. امتیاز جهش درصد جهش یافته هایی  استکه توسط مجموعه آزمون از برنامه ی اصلی تمییز داده می شوند. بدین صورت که این جهش یافته های باعث شکست یک مورد آزمون می شود در حالی که در نسخه ی اصلی مجموعه ی آزمون با موفقیت اجرا می گردد. جهش یافته ها با تزریق خطاهای ساختگی به برنامه تحت آزمون  ساخته می شوند. این خطاهای ساختگی با استفاده از عملگرهای جهش که از پیش تعریف شده اند ساخته می شود. نمونه ی این عملگرها جایگزینی عملگرهای ریاضی یا رابطه ای، تغییر شرط شاخه \endnote{Branch Condition} و یا حذف یک عبارت است\cite{just2014mutants}. تحلیل آزمون در موارد زیر کاربرد دارد:
\begin{itemize}
	\setlength\itemsep{.2cm}	
	\item 
	ارزیابی مجموعه آزمون
	\item 
	انتخاب مجموعه آزمون
	\item 
	 کمینه سازی مجموعه آزمون
	\item 
	 تولید مجموعه آزمون
	\item 
	مکان یابی خطا
	\item 
	پیش بینی خطا
\end{itemize}


