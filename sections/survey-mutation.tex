
\subsection{آزمون جهش و کاربردهای آن}
توسعه دهندگان و پژوهشگران حوزه ی نرم افزار علاقه مند به اندازه گیری موثر بودن مجموعه های آزمون می باشند. توسعه دهندگان به دنبال آن هستند که بدانند مجموعه آزمونهای آنها می تواند به خوبی خطاها را تشخیص دهد و پژوهشگران به دنبال مقایسه ی روشهای مختلف آزمون و اشکال زدایی\endnote{Debugging}  هستند. به طور ایده آل افراد تمایل دارند که بدانند تعداد خطاهایی که یک مجموعه آزمون می تواند شناسایی کند چه مقدار است اما از آنجا که خطاها نا شناخته هستند باید از اندازه گیری وکالتی \endnote{Proxy Measurment} استفاده شود. یکی از این اندازه گیری های شناخته شده امتیاز جهش \endnote{Mutation Score} می باشد که توانایی مجموعه آزمون در تمییز دادن نسخه ی اصلی برنامه از تعداد زیادی نسخه های متفاوت را اندازه گیری می کند. این نسخه های متفاوت که تنها یک تفاوت کوچک نحوی نسبت به برنامه ی اصلی دارند جهش یافته \endnote{Mutant} نامیده می شوند. امتیاز جهش درصد جهش یافته هایی  استکه توسط مجموعه آزمون از برنامه ی اصلی تمییز داده می شوند. بدین صورت که این جهش یافته های باعث شکست یک مورد آزمون می شود در حالی که در نسخه ی اصلی مجموعه ی آزمون با موفقیت اجرا می گردد. جهش یافته ها با تزریق خطاهای ساختگی به برنامه تحت آزمون  ساخته می شوند. این خطاهای ساختگی با استفاده از عملگرهای جهش که از پیش تعریف شده اند ساخته می شود. نمونه ی این عملگرها جایگزینی عملگرهای ریاضی یا رابطه ای، تغییر شرط شاخه \endnote{Branch Condition} و یا حذف یک عبارت است\cite{just2014mutants}. تحلیل آزمون در موارد زیر کاربرد دارد:
\begin{itemize}
	\setlength\itemsep{.01em}	
	\item 
	ارزیابی مجموعه آزمون
	\item 
	انتخاب مجموعه آزمون
	\item 
	 کمینه سازی مجموعه آزمون
	\item 
	 تولید مجموعه آزمون
	\item 
	مکان یابی خطا
	\item 
	پیش بینی خطا
\end{itemize}

روشهایی که از جهش یافته ها به منظور مکان یابی خطا استفاده می کنند دارای شباهت هایی با روشهای پیش بینی خطا هستند. در هر دوی این روشها از معیارهایی  کد منبع استفاده می شود تا احتمال وجود خطا محاسبه شود. دو تفاوت عمده ی این دو حوزه این است که اولا در مکان یابی خطا از روشهای یادگیری ماشین استفاده ی چندانی نمی شود . ثانیا در مکان یابی خطا وجود خطا به وسیله شکست مورد آزمون یا گزارش خطا محرز شده است. با توجه به شباهت های موجود میان این دو حوزه در ادامه چند مقاله که با استفاده از آزمون جهش به مکان یابی خطا پرداخته اند را بررسی می کنیم. \\

"مان" و همکاران در مقاله ی خود بر اساس دو فرض روشی به منظور مکان یابی خطا ارائه داده اند. فرض اول بیان می کند که  در یک برنامه ی حاوی خطا جهش و یا اصلاح یک عبارت خطا دار نسبت به جهش یک عبارت درست می تواند موارد آزمون بیشتری را  با موفقیت بگذراند فرض دوم  بیان می کند که جهش عبارات صحیح نسبت به جهش یک عبارت غلط موجب می شود موارد آزمون بیشتری شکست بخورند. بر اساس این دو فرض معیاری به نام "مشکوک بودن" \endnote{Suspiciousness} ارائه گردیده است که دو فرض را فرموله می کند. این معیار بر اساس تعداد موارد آزمون در نسخه ی اصلی و جهش یافته عمل می کند. سپس با رتبه بندی عبارات بر اساس این معیار عبارت حاوی خطا مشخص می گردد. در این پژوهش روش جدیدی نیز به منظور ارزیابی روش پیشنهادی ارائه شده است که برخی از مشکلات روش پیشین را بر طرف نموده است. در نهایت روش مکان یابی ارائه شده با دو روش ارزیابی شده و نتایج نشان می دهد فرضیات پژوهش درست بوده اند \cite{moon2014ask}. \\

"پاپاداکیس " و "تراوون" در مقاله ی خود به این نکته اشاره کرده اند استفاده از تحلیل جهش در گذشته به دلیل پر هزینه بودن چندان مورد توجه قرار نمی گرفته است اما امروزه با وجود ابزارهای مقیاس پذیر، نمونه گیری و انتخاب جهش می توان به خوبی از تحلیل جهش در انجام پژوهش های مختلف استفاده کرد\cite{papadakis2015metallaxis}. آنها روشی را برای مکان یابی خطا بر اساس دو مشاهده ارائه کرده اند. در مشاهده ی اول دیده می شود که خطای موجود در یک عبارت رفتار مشابهی با جهش در همان عبارت نشان می دهد. در مشاهده ی دیگر دیده می شود که اگر این دو خطا و جهش در دو عبارت متفاوت باشند رفتار متفاوتی خواهند داشت. منظور از رفتار مشابه موفقیت یا شکست در یک آزمون است. بر اساس این دو مشاهده معیاری برای مشکوک بودن عبارات تعیین می گردد. این پژوهش بیان می کند که مناسب بودن موارد آزمون تاثیر مستقیمی بر عملکرد روش مکان یابی خطا را دارد. همچینین یک مجموعه ی کوچک از جهش یافته ها می تواند به اندازه ی مجموعه ای کامل تا ثیر گذار باشد. \\

"جاست" و همکاران در پژوهش خود به بررسی این موضوع پرداخته اند که آیا جهش یافته ها می توانند جایگزین مناسبی برای خطاهای واقعی باشند یا خیر\cite{just2014mutants}. در پژوهش های گذشته بررسی شده بود که میان جهش یافته های ساده و پیچیده وابستگی وجود دارد ولی وابستگی میان جهش یافته های ساده و خطاهای واقعی مشخص نیست. جاست و همکاران دو مجموعه ی آزمون برای هر خطا در نظر گرفتند که مجموعه ی اول در نسخه ی حاوی خطا با موفقیت گذرانده می شود. مجموعه ی دوم در نسخه ی حاوی خطا شکست می خورد و در نسخه ی رفع خطا با موفقیت اجرا می شود. نتایج نشان می دهد که مجموعه ی آزمون دوم دارای امتیاز جهش بالاتری می باشد که نشان می دهد هر خطا به یک جهش یافته وابستگی دارد. لازم به ذکر است که سعی شده که دو مجموعه ی آزمون دارای پوشش یکسانی باشند زیرا پوشش بیشتر می تواند امتیاز جهش بیشتر بیانجامد. همچنین مشخص شد که  
73 \lr{\%} 
 خطاهای واقعی با جهش یافته هایی که  با عملگرهای متدوال تولید شده اند وابستگی دارند. در این پژوهش خطاهایی که با جهش یافته ها وابستگی ندارند در سه دسته قرار می گیرند : دسته اول نیازمند عملگرهای قوی تری هستند. دسته ی دوم نیازمند عملگرهای جدیدی هستند. دسته سوم با جهش یافته ها وابستگی ندارند.\\
 
 "هااو" و همکاران با ارایه ی مجموعه ای از معیارها و استفاده از یادگیری ماشین مدلی را ارائه داده اند که به وسیله ی آن بتوان تشخیص داد علت شکست در آزمون رگرسیون وجود خطا است یا منسوخ \endnote{Obsolete} شدن یک مورد آزمون. هفت معیار ارائه شده در این پژوهش مرتبط با گراف فراخوانی، تغییر در فایل ها و تعداد شکست در آزمونها بوده است.  هااو و همکاران به منظور به دست آوردن مجموعه داده ی حاوی خطا، به صورت دستی بر اساس استانداردهایی  از پیش تعریف شده خطاهایی را در کد قرار داده اند. بدین منظور عباراتی به صورت تصادفی که در سراسر کد محصول قرار دارند انتخاب شدند و به وسیله ی عملگرهای جهش خطاهایی تولید شده است. به منظور بدست آوردن آزمون های منسوخ شده، مجموعه آزمونهایی از نسخه ی قبلی برنامه بر روی کد محصولی نسخه ی بعدی بکار گرفته شده است. سپس با استفاده از روش ارزیابی میان دسته ای به آموزش و آزمایش مدل ساخته شده پرداخته می شود. نتایج پژوهش نشان می دهد که روش پیشنهادی زمانی که بر روی یک نسخه یا نسخه های مختلف از یک برنامه اعمال شود نتایج خوبی دارد (80\lr{\%} دقت) اما زمانی که بر روی برنامه های مختلف اعمال شود ( مجموعه آموزش از یک برنامه و آزمون بر روی برنامه ای دیگر) موثر نیست. نتایج نشان می دهد تکنیک ها مکان یابی خطا نتیجه ی مثبتی بر تشخیص نوع خطا که مربوط به محصول است یا آزمون، ندارد.\\
 
 بوئز و همکاران معیارهایی را مبتنی بر جهش معرفی کردند و مدلی را بر اساس این آنها ساخته اند\cite{bowes2016mutation} معیارهای سنتی و شئ گرایی نیز در ترکیب با آنها مورد استفاده قرار گرفته اند. 8 عملگر جهش در نظر گرفته شده و برای هر یک از آنها یک عملگر ایستا (بدون اجرای کد) و چهار معیار پویا ساخته شده و در مجموعه 40 معیار جهش ارائه شده است. از این جهت میان معیار ایستا و پویا تمایز قائل شده اند که اگر معیارهای ایستا به تنهایی  پیش بینی را بهبود بخشند از آنها بدون نیاز به موارد آزمون می توان استفاده کرد. نتایج پژوهش نشان می دهد که استفاده از معیارهای جهش بهبود قابل توجهی را در پیش بینی خطا به وجود می آورد. همچنین معیارهای پویا و ایستا در کنار یکدیگیر توانایی پیش بینی مناسبی دارند ولی استفاده ی جداگانه از آنها تاثیر چندان مثبتی نخواهد داشت. این پژوهش از دو جنبه حائز اهمیت می باشد. یکی اینکه اولین پژوهش در زمینه ی پیش بینی خطاست که از تحلیل جهش استفاده کرده است. دوم آنکه مشابه ترین پژوهش به پژوهش کنونی می باشد. 
 
 
 
 
 
 
 
 
 
 
 
 
 
 
 
 
 
 
 
 
 

