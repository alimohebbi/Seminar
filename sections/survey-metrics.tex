\subsubsection{معیارهای پیش بینی خطا}

معیارهای پیش بینی خطا نقش مهمی را در ساخت مدل پیش بینی ایفا می کنند. اکثریت معیارهای پیش بینی خطا را می توان به دو دسته ی تقسیم کرد: معیارهای کد و معیارهای فرآیند. معیارهای کد می توانند به طور مستقیم از کدهای منبع موجود جمع آوری شوند در حالی که معیارهای فرآیند  از اطلاعات تاریخی که در مخازن نرم افزاری مختلف آرشیو شده اند. نمونه از این مخازن نرم افزاری سیستم های کنترل نسخه و سیستم های ردگیری خطا است. معیار های فرآیند از نظر هزینه موثر تر از سایر معیارها هستند\cite{arisholm2010systematic}. در برخی از مقالات نیز معیارهای  پیش بینی خطا به سه دسته ی: معیارهای کد منبع سنتی، معیارهای شئ گرایی، معیارهای فرآیند تقسیم شده اند\cite{radjenovic2013software}.\\

معیارهای کد تحت عنوان معیارهای محصول 
\endnote{Product Metrics}
نیز شناخته می شوند و میزان پیچیدگی کد را می سنجند. فرض زمینه ای آنها این است که هرچه کد پیچیده تر باشد خطا خیز تر است. برای اندازه گیری پیچیدگی کد پژوهشگران معیار های مختلفی را ارائه داده اند که در ادامه به معرفی مهم ترین آنها پرداخته خواهد شد. معیارهای "اندازه" اندازه ی کلی و حجم کد را می سنجند. نماینده ی این معیارهای "تعداد خطوط" می باشد. اولین بار توسط آکیاما  \cite{akiyama1971example}  ارائه شد. هالستد  \cite{halstead1977elements} چندین معیار اندازه بر اساس  تعداد عملگرها و عملوند ها ارائه داده است. مک کیب معیارهای "حلقه ای" 
\endnote{Cyclomatic}
را پیشنهاد داد که این معیار با استفاده از تعداد گره ها، یالها و قطعات متصل در گراف جریان کنترلی کد منبع محاسبه می گردد\cite{mccabe1976complexity}. این معیارها نشان می دهند که راه های کنترلی به چه میزان پیچیده هستند. با ظهور زبانهای شئ گرایی و محبئبیت آنها معیارهای کد معیارهای کد برای این زبانها ارائه شد تا فرآیند توسعه بهبود یابد. نماینده ی معیارهای شی گرایی "چدامر و کمر" (\lr{CK}) می باشند\cite{chidamber1994metrics}. این معیارها طراحی شدند تبا توجه به خصیصه های زبانهای شئ گرایی مانند وارثت، زوجیت، همبستگی طراحی شده اند. بجز معیارهای \lr{CK} معیارهای شئ گرایی دیگری نیز بر اساس حجم و کمیت کد منبع پیشنهاد داده شده اند. همانند معیارهای "اندازه" معیارهای شئ گرایی تعداد نمونه های یک کلاس، توابع و غیره را می شمارند. \\
در ادامه به بررسی تعداد از معیارهای فرآیند پدراخته می شود که در این دسته شاخص محسوب می شوند. "ناگاپان و بال" 8 معیار تغییر نسبی کد را ارائه داده اند\cite{nagappan2005use}. به عنوان مثال یکی از معیار های آنها تعداد تجمعی خطوط اضافه و حذف شده بین دو نسخه از برنامه را می شمارد و بر تعداد خطوط برنامه تقسیم می کند. معیار دیگر تعداد فایلهای تغییر یافته از یک قطعه برنامه را بر تعداد فایلها تقسیم می کند. دسته ی دیگری از معیارهای فرآیند معیار تغییر هستند این معیار به عنوان مثال تعداد رفع خطاها، تعداد بازآرایی کد \endnote{Refactoring} و یا تعداد نویسنگان یک فایل را می شمارند. "موزر" و همکاران معیارهایی را ارائه داده اند که تعداد خطوط اضافه و کم شده را بدون در نظر گرفتن تعداد کل خطوط می شمارد. در عوض سن فایل ها و تعداد فایلهایی که کامیت می شوند در نظر گرفته می شود \cite{moser2008comparative}.  "بچلی" و همکاران معیارهای محبوبیت را بر اساس تحلیل رایانامه های آرشیو شده ی نویسنگان ارائه داده اند. ایده ی اصلی این معیارها این است که یک قطعه ی  نرم افزاری که در رایانامه ها درباره ی آن بیشتر صحبت شده است خطا خیز تر می باشد\cite{bacchelli2010popular}. "برد" و همکاران چهار معیار مالکیت بر اساس نویسندگان یک قطعه ارائه داده اند. مالکیت یک قطعه بر اساس نسبت کامیت هایی که افراد (مشارکت آنها) در یک قطعه  تعریف می شود. \\

"راجنویک" و همکاران در پژوهش خود به بررسی قاعده مند \endnote{Systematic Review} معیارهای پیش بینی خطا در مطالعات پیشین پرداخته اند.  طبق این پژوهش در 49\lr{\%} مطالعات از معیارهای شی گرایی، در 27 \lr{\%} معیارهای سنتی کد و در 26 \lr{\%} از معیارهای فرآیند استفاده شده است. با توجه به مطالعات بررسی شده دقت پیش بینی خطا تفاوت قابل توجهی با انتخاب معیارهای مختلف پیدا می کند. معیارهای شی گرایی و فرآیند موفق تر از معیارهای سنتی هستند. معیارهای سنتی پیش بینی قویا با معیارهای اندازه مانند تعداد خطوط کد همبستگی دارند و این دو توانایی پیش بینی خطا دارند اما جز بهترین معیارها نیستند. معیارهای شی گرایی بهتر از اندازه و پیچیدگی عمل می کنند و با این که با معیارهای اندازه همبستگی دارند اما ویژگی های بیشتری علاوه بر اندازه را دارند. معیارهای ایستای کد همانند اندازه، پیچیدگی و شئ گرایی به منظور بررسی یک نسخه از برنامه مفید هستند اما زمانی که با هر تکرار در فرآیند توسعه ی نرم افزار دقت پیش بینی آنها کاسته می شوند و معیارهای فرآیند در چنین شرایطی بهتر عمل می کنند. علارغم توانمندی بالقوه، معیارهای فرآیند در تعداد کمتری از پژوهش ها مورد استفاده قرار گرفته اند \cite{radjenovic2013software}. \\
 
آسترند و همکاران به بررسی این موضوع پرداختند که آیا اطلاعات درباره ی اینکه کدام توسعه دهنده یک فایل را اصلاح می کند قادر است که پیش بینی خطا را بهبود بخشد. در پژوهش قبلی آنها\cite{weyuker2008too} مشخص شده بود که تعداد کلی افراد توسعه دهنده در یک فایل می تواند در پیش بینی خطا تاثیر متوسطی داشته باشد. در این پژوهش تعدادی از متغیرهای کد منبع و فرآیند به همراه معیار مرتبط به توسعه دهنده در نظر گرفته شده است. تعداد خطاهایی که یک توسعه دهنده تولید می کند ثابت است و با سایر توسعه دهندگان فرق دارد. این تفاوت با در نظر گرفتن حجم کدی که یک توسعه دهنده اصلاح می کند مرتبط است و در نتیجه در نظر گرفتن یک نویسنده خاص نمی تواند به بهبود پیش بینی خطا کمک کند\cite{ostrand2010programmer}. \\

رحمان و دوانبو از جنبه های مختلف معیارهای فرآیند  را با سایر معیارها مقایسه کرده اند\cite{rahman2013and}. نتایج نشان می دهد  زمانی که مدل پیش بینی بر روی یک نسخه آموزش می بیند و در نسخه ی بعدی آزموده می شود معیارهای کد \lr{AUC} قابل قبولی دارند اما کمتر از معیارهای فرآیند و از نظر معیار 
\lr{AUCEC 20\%} 
بهتر از یک مدل تصادفی عمل نمی کند و این بدان معنی است که این معیارهعا از نظر هزینه چندان  موثر نیستند. همچنین معیارهای کد ایستا تر هستند، به این معنی که یعنی با تغییرات پروژه و تغییر در توزیع خطاها همچنان معیارها بدون تغییر باقی بمانند. معیار ایستا تمایل دارد یک فایل را انتشار\endnote{Release} پس از انتشار همچنان حاوی خطا معرفی کند. معیارهای ایستا به مدلهای راکد منجر می شوند که این مدلهابه سمت فایل های بزرگ با تراکم خطای کمتر بایاس دارند. به عنوان مثال حالتی را در نظر بگیرید که در یک پروژه فایلهای بزرگ و پیچیده ای وجود دارد که پس از چندین انتشار خطهای آنها برطرف می شود اما مدلهایی که بر اساس معیارهای کد ساخته شده اند همچنان این فایلها را به عنوان خطا خیز معرفی می کنند. از طرف دیگر حالتی را در نظر بگیرید که یک فایل با اندازه و پیچیدگی کم به تازگی به وجود آمده و یا تغییرات فراوان یافته است. مدلهای مبتنی بر کد به این فایلها توجه چندانی نخواهند کرد در حالیکه که این فایل ها مستعد وجود خطا هستند. بدین ترتیب معیارهای فرآیند بهتر از معیارهای کد عمل می کنند. 
 
 
