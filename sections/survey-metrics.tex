\subsubsection{معیارهای پیش بینی خطا}

معیارهای پیش بینی خطا نقش مهمی را در ساخت مدل پیش بینی ایفا می‌کنند. اکثریت معیارهای پیش بینی خطا را می‌توان به دو دسته‌ی  کلی تقسیم کرد: معیارهای کد و معیارهای فرآیند. معیارهای کد می‌توانند به طور مستقیم از کدهای منبع موجود جمع آوری شوند در حالی که معیارهای فرآیند  از اطلاعات تاریخی که در مخازن نرم افزاری مختلف آرشیو شده‌اند استخراج می‌گردند. نمونه‌ای از این مخازن نرم افزاری سیستم‌های کنترل نسخه و سیستم‌های ردگیری خطا است. معیار‌های فرآیند از نظر هزینه موثر تر از سایر معیارها هستند\cite{arisholm2010systematic}. در برخی از مقالات نیز معیارهای  پیش بینی خطا به سه دسته‌ی: معیارهای کد منبع سنتی، معیارهای شئ گرایی و معیارهای فرآیند تقسیم شده‌اند\cite{radjenovic2013software}.\\

معیارهای کد تحت عنوان معیارهای محصول\endnote{Product Metrics}
نیز شناخته می‌شوند و میزان پیچیدگی کد را می‌سنجند. فرض زمینه‌ای\endnote{Ground Assumption} آنها این است که هرچه کد پیچیده تر باشد خطا خیز تر است. برای اندازه گیری پیچیدگی کد پژوهشگران معیار‌های مختلفی را ارائه داده‌اند که در ادامه به معرفی مهم ترین آنها پرداخته خواهد شد. معیارهای "اندازه" اندازه‌ی کلی و حجم کد را می‌سنجند. نماینده‌ی این معیارها "تعداد خطوط" می‌باشد و اولین بار توسط آکیاما  \cite{akiyama1971example}  ارائه شد. هالستد  \cite{halstead1977elements} چندین معیار اندازه بر اساس  تعداد عملگرها و عملوند‌ها ارائه داده است. مک کیب معیارهای "حلقه‌ای" 
\endnote{Cyclomatic}
را پیشنهاد داد که این معیار با استفاده از تعداد گره‌ها، یالها و قطعات متصل در گراف جریان کنترلی\endnote{Control Flow}  کد منبع محاسبه می‌گردد\cite{mccabe1976complexity}. این معیارها نشان می‌دهند که راه‌های کنترلی به چه میزان پیچیده هستند. با ظهور زبانهای شئ گرایی و محبوبیت آنها معیارهای کد  برای این زبانها ارائه شد تا فرآیند توسعه بهبود یابد. نماینده‌ی معیارهای شی گرایی "چدامر و کمر" (\lr{CK}) می‌باشند\cite{chidamber1994metrics}. این معیارها با توجه به خصیصه‌های زبانهای شئ گرا مانند وارثت، زوجیت، همبستگی\endnote{Cohesion} طراحی شده‌اند. بجز معیارهای \lr{CK} معیارهای شئ گرایی دیگری نیز بر اساس حجم و کمیت کد منبع پیشنهاد داده شده‌اند. مشابه معیارهای "اندازه"، معیارهای شئ گرایی تعداد نمونه‌های یک کلاس، توابع و غیره را می‌شمارند. \\
در ادامه به بررسی تعدادی از معیارهای فرآیند پرداخته می‌شود که در این دسته شاخص محسوب می‌شوند. ناگاپان و بال 8 معیار تغییر نسبی کد را ارائه داده‌اند\cite{nagappan2005use}. به عنوان مثال یکی از معیار‌های آنها تعداد تجمعی خطوط اضافه و حذف شده بین دو نسخه از برنامه را می‌شمارد و بر تعداد خطوط برنامه تقسیم می‌کند. معیار دیگر تعداد فایلهای تغییر یافته از یک قطعه برنامه را بر تعداد  کل فایلها تقسیم می‌کند. دسته‌ی دیگری از معیارهای فرآیند معیار تغییر هستند این معیار به عنوان مثال تعداد رفع خطاها، تعداد بازآرایی کد \endnote{Refactoring} و یا تعداد نویسنگان یک فایل را می‌شمارند. موزر و همکاران معیارهایی را ارائه داده‌اند که تعداد خطوط اضافه و کم شده را بدون در نظر گرفتن تعداد کل خطوط می‌شمارد. در عوض سن فایل‌ها و تعداد فایلهایی که کامیت می‌شوند در نظر گرفته می‌شود \cite{moser2008comparative}.  بچلی و همکاران معیارهای شهرت\endnote{Popularity} را بر اساس تحلیل ایمیل‌های آرشیو شده‌ی نویسنگان ارائه داده‌اند. ایده‌ی اصلی این معیارها این است که یک قطعه‌ی  نرم افزاری که در ایمیل‌ها درباره‌ی آن بیشتر صحبت شده است خطا خیز تر می‌باشد\cite{bacchelli2010popular}. برد و همکاران چهار معیار مالکیت بر اساس نویسندگان یک قطعه ارائه داده‌اند. مالکیت یک قطعه بر اساس نسبت کامیت‌های  افراد  در یک قطعه (مشارکت آنها) تعریف می‌شود. \\

راجنویک و همکاران در پژوهش خود به بررسی قاعده مند \endnote{Systematic Review} معیارهای پیش بینی خطا در مطالعات پیشین پرداخته‌اند.  طبق این پژوهش در 49\lr{\%} مطالعات از معیارهای شی گرایی، در 27 \lr{\%} معیارهای سنتی کد و در 26 \lr{\%} از معیارهای فرآیند استفاده شده است. با توجه به مطالعات بررسی شده دقت پیش بینی خطا  با انتخاب معیارهای مختلف، تفاوت قابل توجهی  پیدا می‌کند. معیارهای شئ گرایی و فرآیند موفق تر از معیارهای سنتی هستند. معیارهای سنتی  پیچیدگی کد، قویا با معیارهای اندازه مانند تعداد خطوط کد همبستگی دارند و این دو توانایی پیش بینی خطا دارند اما جز بهترین معیارها نیستند. معیارهای شی گرایی بهتر از اندازه و پیچیدگی عمل می‌کنند و با این که با معیارهای اندازه همبستگی دارند اما ویژگی‌های بیشتری علاوه بر اندازه را دارند. معیارهای ایستای کد همانند اندازه، پیچیدگی و شئ گرایی به منظور بررسی یک نسخه از برنامه مفید هستند اما با هر تکرار\endnote{Iteration} در فرآیند توسعه‌ی نرم افزار دقت پیش بینی آنها کاسته می‌شوند و معیارهای فرآیند در چنین شرایطی بهتر عمل می‌کنند. علارغم توانمندی بالقوه، معیارهای فرآیند در تعداد کمتری از پژوهش‌ها مورد استفاده قرار گرفته‌اند \cite{radjenovic2013software}. \\
 
آسترند و همکاران به بررسی این موضوع پرداختنه‌اند که آیا اطلاعاتی درباره‌ی اینکه کدام توسعه دهنده یک فایل را اصلاح می‌کند قادر است که پیش بینی خطا را بهبود بخشد. در پژوهش قبلی آنها\cite{weyuker2008too} مشخص شده بود که تعداد کلی   افراد توسعه دهنده در یک فایل می‌تواند در پیش بینی خطا تاثیر متوسطی داشته باشد. در  مقاله‌ی \cite{ostrand2010programmer}  تعدادی از متغیرهای کد منبع و فرآیند به همراه معیار مرتبط به توسعه دهنده در نظر گرفته شده است.  در این پژوهش مشخص شد  که تعداد خطاهایی که یک توسعه دهنده تولید می‌کند ثابت است و با سایر توسعه دهندگان فرق دارد. این تفاوت با  حجم کدی که یک توسعه دهنده اصلاح می‌کند مرتبط است و در نتیجه در نظر گرفتن یک نویسنده خاص نمی‌تواند به بهبود پیش بینی خطا کمک کند\cite{ostrand2010programmer}. \\

رحمان و دوانبو از جنبه‌های مختلف معیارهای فرآیند  را با سایر معیارها مقایسه کرده‌اند\cite{rahman2013and}. نتایج نشان می‌دهد  زمانی که مدل پیش بینی بر روی یک نسخه آموزش می‌بیند و در نسخه‌ی بعدی آزموده می‌شود معیارهای کد، \lr{AUC} قابل قبولی دارند اما  \lr{AUC}  آنها کمتر از معیارهای فرآیند است  و از نظر معیار 
\lr{AUCEC 20\%} 
بهتر از یک مدل تصادفی عمل نمی‌کنند و  به آن معنی است که این معیارها از نظر هزینه چندان  موثر نیستند. همچنین معیارهای کد ایستاتر هستند، ‌یعنی با تغییرات پروژه و تغییر در توزیع خطاها همچنان معیارها بدون تغییر باقی می‌مانند. معیار ایستا تمایل دارد یک فایل را انتشار\endnote{Release} پس از انتشار همچنان حاوی خطا معرفی کند. معیارهای ایستا به مدلهای راکد منجر می‌شوند که این مدلها به سمت فایل‌های بزرگ با تراکم خطای کمتر بایاس دارند. به عنوان مثال حالتی را در نظر بگیرید که در یک پروژه فایلهای بزرگ و پیچیده‌ای وجود دارد که پس از چندین انتشار خطهای آنها برطرف می‌شود اما مدلهایی که بر اساس معیارهای کد ساخته شده‌اند همچنان این فایلها را به عنوان خطا خیز معرفی می‌کنند. از طرف دیگر حالتی را در نظر بگیرید که یک فایل با اندازه و پیچیدگی کم به تازگی به وجود آمده و یا تغییرات فراوان یافته است. مدلهای مبتنی بر کد به این فایلها توجه چندانی نخواهند کرد در حالیکه که این فایل‌ها مستعد وجود خطا هستند. بدین ترتیب معیارهای فرآیند بهتر از معیارهای کد عمل می‌کنند. 
 
 
